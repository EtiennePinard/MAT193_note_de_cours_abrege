\unnumsec{Chapitre 7: Opérateurs inverses, hermitiens et unitaires}

\subsection{Opérateurs inverses}
Soit $T\colon V \to W$ un opérateur linéaire
\begin{definition}
    $T$ inversible $\iff \forall \ w \in W \ \exists! \ v \in V$ t.q. $Tv = w$
\end{definition}
\begin{definition}
    L'opérateur inverse de $T$ est $\myfunc{T^{-1}}{W}{V}{w}{v}$ 
\end{definition}
Soit $B_V, B_W$ des bases dans $V, W$ respectivement 
\begin{theorem}
    $T$ inversible $\iff \dim V = \dim W$ et $[T]_{B_V, B_W}$ inversible \\
    Dans ce cas, $T^{-1} = [T^{-1}]_{B_W, B_V} = \left([T]_{B_V, B_W}\right)^{-1}$
\end{theorem}
\begin{corollary}
    $T$ inversible $\iff [T]_{B_V, B_W}$ carré
\end{corollary}
\begin{remark}
    $[T]_{B_V, B_W}$ carré $\implies \dim V = \dim W$
\end{remark}

\subsubsection{Propriétés de l'opérateur inverse}
Soit $T, S\colon V \to W$ des opérateurs linéaires inversibles
\begin{enumerate}
    \item $T^{-1}\colon W \to V$ est linéaire
    \item $[T^{-1}]_{B_W, B_V} = \left([T]_{B_V, B_W}\right)^{-1}$
    \item $\left(T^{-1}\right)^{-1} = T$
    \item $(TS)^{-1} = S^{-1}T^{-1}$
    \item $T$ inversible $\iff \text{Ker}(T) = \{0_W\}$
\end{enumerate}

\subsection{Opérateurs hermitiens}
Soit $V$ un espace vectoriel muni d'un produit scalaire $\scpr{\cdot}{\cdot}$ \\
Soit $B$ une base orthonormale dans $V$ par rapport au produit scalaire dans $V$ \\
Soit $A\colon V \to V$ un opérateur linéaire
\begin{definition}
    $A$ hermitien $\iff [A]_B$ hermitien 
\end{definition}
\begin{theorem}
    Peut importe la base $B$ choisi, si $A$ hermitien alors $[A]_B$ hermitien
\end{theorem}
\begin{remark}
    Le théorème nous permet de justifier la définition d'un opérateur hermitien, puisque 
    le théorème prouve que la définition ne dépend pas de la base choisi
\end{remark}

\subsubsection{Propriétés de l'opérateur hermitien}
Soit $A, T\colon V \to V$. Les opérateurs suivants sont hermitiens
\begin{align*}
    1. \quad A + T& &2. \quad \alpha A \ \text{si} \ \alpha \in \R& &3. \quad AT \ \text{si} \ A,T \ \text{commute}& &4. \quad A^n, T^n& &5. \quad A^{-1} \ \text{si} \ A \ \text{inversible} 
\end{align*}
De plus, un opérateur hermitien est indépendant dans le produit scalaire. 
\begin{lemma}
    Soit $v_1, v_2 \in V$, alors $\scpr{v_1}{Av_2}_V = \scpr{Av_1}{v_2}_V$
\end{lemma}
\noindent
Dans la notation de Dirac, on note cette propriété comme tel: $\bra{v_1}A\ket{v_2}$,
ce qui indique que $A$ peut être appliqué à $\bra{v_1}$ ou $\ket{v_2}$

\subsection{Opérateur unitaire}
Soit $V$ un espace vectoriel muni d'un produit scalaire $\scpr{\cdot}{\cdot}$ \\
Soit $B$ une base orthonormale dans $V$ par rapport au produit scalaire dans $V$ \\
Soit $A\colon V \to V$ un opérateur linéaire
\begin{definition}
    $A$ unitaire $\iff [A]_B$ unitaire 
\end{definition}
\begin{remark}
    Comme dans le cas de l'opérateur hermitien, cette définition ne dépend pas de la base $B$ choisi
\end{remark}

\subsubsection{Propriétés de l'opérateur unitaire}
\begin{lemma}
    Si $A\colon V \to V$ unitaire et $v_1, v_2 \in V$, alors $\scpr{v_1}{v_2}_V = \scpr{Av_1}{Av_2}_V $
\end{lemma}