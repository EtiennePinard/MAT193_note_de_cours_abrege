\documentclass[11pt]{article}

\usepackage[T1]{fontenc}
\usepackage[french]{babel}
\usepackage[margin = 2cm]{geometry}
\usepackage[shortlabels]{enumitem}
\usepackage[hidelinks]{hyperref}
\usepackage{math_and_phy_utils}
\usepackage[pdftex]{graphicx}
\usepackage{tikz}
\usetikzlibrary{angles,quotes, matrix,arrows,decorations.pathmorphing}

\usepackage{fancyhdr}
\setlength{\headheight}{15.2pt}
% The footnote code is from https://tex.stackexchange.com/a/74992
\newcommand{\fancyfootnotetext}[2]{%
  \fancypagestyle{dingens}{%
    \fancyfoot[C]{}
    \fancyfoot[L]{\footnotemark[#1]\footnotesize #2}%
    \fancyfoot[R]{\thepage}
  }%
  \thispagestyle{dingens}%
}

\pagestyle{fancy}
\fancyhead{}
\fancyfoot[c]{\thepage}

\usepackage{amsmath}
\usepackage{amssymb}
\usepackage{amsthm}

\theoremstyle{plain}
\newtheorem{theorem}{Théorème}[section]
\newtheorem{lemma}[theorem]{Lemme}
\newtheorem{corollary}[theorem]{Corollaire}

\theoremstyle{definition}
\newtheorem{definition}[theorem]{Définition}

\theoremstyle{remark}
\newtheorem*{remark}{Remarque}
\newtheorem*{note}{Note}
\AfterEndEnvironment{note}{\noindent\ignorespaces}
\AfterEndEnvironment{remark}{\noindent\ignorespaces}

\newcommand{\unnumsec}[1]{
    \refstepcounter{section}
    \addcontentsline{toc}{section}{Chapitre \thesection: #1}
    \fancyhead[c]{Chapitre \thesection: #1}
    \section*{Chapitre \thesection: #1}
}

\begin{document}

% This is to disable the spacing before a colon of the french babel package. Use \shorthandon{:} to renable it
\shorthandoff{:}

% Gotten from https://tex.stackexchange.com/a/325032
\begin{titlepage}
  \centering
  \vspace*{3in}

  \vspace*{0.5cm}

  \Huge
  \textmd{\textbf{Algèbre linéaire}}\\
  \vspace{0.1in} \LARGE \textmd{\textbf{Notion de base}} \\
  \vspace{0.1in}\large{Basé sur le cours MAT193}

  \vspace*{\fill}
  \large Étienne Pinard \\
  Compilé le \today

\end{titlepage}

\pagebreak

\fancyhead[c]{Table des matières}
\tableofcontents

\unnumsec{Nombres complexes}


\subsection{Définition des nombres complexes}
Considérons la ligne des nombres réels
\begin{figure}[!htbp]
    \begin{center}
        \def\numberLineLastNumber{7}
        \begin{tikzpicture}
            \begin{scope}[thick,font=\scriptsize]

                \draw [<->] (-\numberLineLastNumber - 1,0) -- (\numberLineLastNumber + 1,0);

                \foreach \n in {-\numberLineLastNumber,...,\numberLineLastNumber}{%
                        \draw (\n,-3pt) -- (\n,3pt)   node [above] {$\n$};
                    }
            \end{scope}
        \end{tikzpicture}
    \end{center}
    \caption{La ligne des nombres réels}
\end{figure}
\begin{remark}
    Chaque nombre réel peut être représenté par un vecteur qui commence de 0 et qui a le nombre comme magnitude. L'angle du vecteur est encodé dans le signe du nombre, un nombre positif va avoir un vecteur d'un angle de 0 et un nombre négatif va avoir un vecteur d'un angle de $\pi$
\end{remark}

\begin{figure}[!htbp]
    \begin{center}
        \def\numberLineLastNumber{7}
        \begin{tikzpicture}
            \begin{scope}[thick,font=\scriptsize]

                \draw [<->] (-\numberLineLastNumber - 1,0) -- (\numberLineLastNumber + 1,0);

                \foreach \n in {-\numberLineLastNumber,...,\numberLineLastNumber}{%
                        \draw (\n,-3pt) -- (\n,3pt)   node [above] {$\n$};
                    }

                \draw [-, thick, color=red] (0,0) -- (-3,0);
                \draw [-, thick, color=green] (0,0) -- (7,0);
                \draw [color=black, fill=black] (-3,0) circle(0.075);
                \draw [color=black, fill=black] (7,0) circle(0.075);

            \end{scope}
        \end{tikzpicture}
    \end{center}
    \caption{Le vecteur des nombres -3 et 7 représenté sur la ligne des nombres réels}
\end{figure}
\begin{remark}
    Dans cette représentation, les vecteurs ont seulement des angles de 0 ou $\pi$. Pour que les vecteurs aillent des angles arbitraires, nous devons introduire un autre axe.
\end{remark}
\begin{definition}
    $i$ est l'unité imaginaire telle que $i^2 = -1$
\end{definition}
Comme dans le cas des nombres réels, il est possible de tracer une ligne des nombres imaginaires. Par convention, cette ligne est tracée perpendiculaire à la ligne des nombres réels. Ainsi, la ligne des nombres imaginaires est justement ce qu'il faut pour pouvoir représenter des vecteurs avec des angles arbitraires.
\begin{figure}[!htbp]
    \begin{center}
        \begin{tikzpicture}
            \begin{scope}[thick,font=\scriptsize]

                \draw [->] (-3.5,0) -- (4,0) node [above]  {Axe réel};
                \draw [->] (0,-3.5) -- (0,4) node [below right] {Axe imaginaire};

                \foreach \n in {-3,...,-1,1,2,...,3}{%
                        \draw (\n,-3pt) -- (\n,3pt)   node [above] {$\n$};
                        \draw (-3pt,\n) -- (3pt,\n)   node [right] {$\n i$};}

                \draw [thick, color=red] (0,0) -- (2,2);
                \draw [color=black, fill=black] (2,2) circle(0.05);
                \node [color=black] at (2,2.25) {$ 2+2i$};
            \end{scope}
        \end{tikzpicture}
    \end{center}
    \caption{Représentation du vecteur qui a une magnitude de $2\sqrt{2}$ et un angle de $\frac{\pi}{4}$}
    \label{complex_plane}
\end{figure}

\begin{definition}
    L'ensemble qui est représenté par la ligne des nombres réels et la ligne des nombres imaginaires est appelé les nombres complexes et est noté $\C$
\end{definition}

\subsection{Représentation des nombres complexes}
Considérons la figure suivante:
\begin{figure}[!htbp]
    \begin{center}
        \begin{tikzpicture}
            \begin{scope}[thick,font=\scriptsize]
                \coordinate (start) at (4,0);
                \coordinate (middle) at (0,0);
                \coordinate (end) at (2, 3);

                \draw [->] (-0.5,0) -- (4,0) node [above]  {Axe réel};
                \draw [->] (0,-0.5) -- (0,4) node [below right] {Axe imaginaire};

                \draw [thick, color=red] (0,0) -- (2,3);
                \node [color=black] at (0.88, 1.62) {$r$};
                \pic [draw, -, "$\theta$", angle eccentricity=1.5] {angle = start--middle--end};

                \draw [color=blue, fill=blue] (2,3) circle(0.05);
                \node [color=black] at (2,3.25) {$z \in \C$};

                \draw [thick, color=green] (0,0) -- (2, 0);
                \node [color=black] at (1, -0.25) {$x$};

                \draw [thick, color=green] (2,0) -- (2, 3);
                \node [color=black] at (2.25, 1.5) {$y$};

            \end{scope}
        \end{tikzpicture}
    \end{center}
    \caption{Représentation d'un nombre complexe}
\end{figure}

\begin{definition}
    $x$ est la partie réelle de $z$ et est notée $\Real{z}$
\end{definition}
\begin{definition}
    $y$ est la partie imaginaire de $z$ et est notée $\Ima{z}$
\end{definition}
\begin{definition}
    $r$ est le module de $z$ et est noté $|z|$
\end{definition}
\begin{lemma} On peut calculer le module de $z$ à partir de sa partie réelle et imaginaire avec
    \[|z| = r = \sqrt{x^2 + y^2}\]
\end{lemma}
\begin{definition}
    $\theta$ est l'argument de $z$ noté $\text{Arg}(z)$. Cet argument est unique et par convention est entre $-\pi$ et $\pi$, soit $-\pi < \text{Arg}(z) \leq \pi$.
\end{definition}
\begin{definition}
    L'argument non unique de $z$ est $\arg(z) = \text{Arg}(z) + 2\pi k, \ k \in \Z$
\end{definition}
\begin{lemma}
    On peut calculer l'argument de $z$ à partir de sa partie réelle et imaginaire avec
    \[\textup{Arg}(z) = \theta =
        \begin{cases}
            \arctan(\frac{y}{x})          & x > 0                                                  \\
            \arctan(\frac{y}{x}) - \pi    & x < 0                                                  \\
            \textup{sign}(y)\frac{\pi}{2} & x = 0, \ \textup{sign}(y) \ \textup{est la signe de y}
        \end{cases}\]
\end{lemma}
On peut remarquer qu'un nombre complexe est entièrement déterminé par sa partie réelle et imaginaire. On peut représenter un nombre complexe de trois formes différentes. Si on utilise seulement $x$ et $y$ pour représenter ces deux parties, on obtient la première forme d'un nombre complexe:
\begin{definition}
    La \underline{forme algébrique} (ou cartésienne) de $z$ est $z = x + iy, \ x, y \in \R$
\end{definition}
Si on utilise seulement $r$ et $\theta$ pour représenter la partie réelle et imaginaire, on obtient la deuxième forme d'un nombre complexe:
\begin{definition}
    La \underline{forme polaire} de $z$ est $z = r(\cos(\theta)+i\sin(\theta)), \ r, \theta \in \R$
\end{definition}
Il existe une troisième forme d'un nombre complexe. Cette forme utilise la formule d'Euler, qui est:
\begin{theorem}[Formule d'Euler]
    \[ e^{i\theta} = \cos{\theta} + i\sin{\theta} \]
\end{theorem}
En utilisant la formule d'Euler on obtient la forme exponentielle:
\begin{definition}
    La \underline{forme exponentielle} de $z$ est $z = re^{i\theta}, \ r, \theta \in \R$
\end{definition}
\begin{note}
    Les trois formes sont équivalentes, c-à-d $z = x + iy = r(\cos{\theta} + i\sin{\theta}) = re^{i\theta}$
\end{note}

\begingroup
\renewcommand{\arraystretch}{1.5}

\subsection{Le conjugué d'un nombre complexe}
\begin{definition}
    Le conjugé de $z = x + iy$ est noté $z^*$ ou $\bar{z}$ et est donné par $z^* = x - iy = \Real{z} - i\Ima{z}$
\end{definition}
\begin{remark}
    Le conjugué est involutif, c-à-d $\left(z^* \right)^* = z$
\end{remark}

\subsubsection{Propriétés du conjugué et du module}
Le conjugué et le module sont distributifs sur l'addition, la multiplication et la division.
\begin{center}
    \begin{tabular}{c@{\hskip 1in} c@{\hskip 1in} c}
        Addition                                   & Multiplication                          & Division                                                  \\
        $\left(z_1 + z_2\right)^* = z_1^* + z_2^*$ & $\left(z_1z_2\right)^* = z_1^{*} z_2^*$ & $\left( \frac{z_1}{z_2}\right)^{*} = \frac{z_1^*}{z_2^*}$ \\
        $\left|z_1 + z_2\right| = |z_1| + |z_2|$   & $\left|z_1z_2\right| = |z_1||z_2|$      & $\left| \frac{z_1}{z_2}\right| = \frac{|z_1|}{|z_2|}$
    \end{tabular}
\end{center}
Le conjugué et le module sont reliés par l'équation
\[
    zz^* = |z|^2
\]
\subsubsection{Propriétés de l'argument}
L'argument de $z$ a des propriétés semblables aux propriétés logarithmiques
\begin{center}
    \begin{tabular}{c@{\hskip 1in}c}
        Produit à somme                        & Quotient à différence                                       \\
        $\arg(z_1z_2) = \arg(z_1) + \arg(z_2)$ & $\arg\left( \frac{z_1}{z_2}\right) = \arg(z_1) - \arg(z_2)$
    \end{tabular}
\end{center}

\endgroup

\subsection{Racines entières}
\label{whole_roots}
Soit $z = re^{i\theta},\ n \in \N$. Alors $z$ à la racine de $n$ est donné par
\[
    \sqrt[n]{z} = \sqrt[n]{r} \left( e^{i \left( \frac{\theta + 2\pi k}{n} \right) } \right), \ k = 0, \dots, n - 1
\]
On peut aussi calculer les racines de $z$ de manière récursive. \\
Si $w$ est une racine de $z$, alors on a
\[
    w_k = \begin{cases}
        w_{k - 1}e^{ \frac{2\pi i}{n} }                    & k > 0 \\[0.5em]
        \sqrt[n]{r} \left( e^{i \frac{\theta}{n} } \right) & k = 0
    \end{cases}
\]
\begin{remark}
    Deux racines entières consécutives sont séparées par un angle de $\frac{2\pi}{n}$
\end{remark}

\subsection{Exponentielle  et logarithme}
Soit $z \in \C$, alors l'exponentielle de $z$ noté $e^z$ est
\[
    e^z = e^x e^{iy} = e^x (\cos(y) + i\sin(y))
\]
De plus, le logarithme de $z$ noté $\ln{z}$ ou $\log{z}$ est
\begin{align*}
    \ln(z) = \log(z) & = \ln|z| + i(\theta + 2\pi k), \ k \in \Z        \\
                     & = \ln|z| + i(\text{Arg}(z) + 2\pi k), \ k \in \Z \\
                     & = \ln|z| + i\arg(z)                              \\
\end{align*}

\subsection{Puissances complexes}
Soient $z, w \in \C$ où $w = x + iy$, alors $z$ à la $w$ est donnée par
\[
    z^w = |z|^{x} e^{-y\text{Arg}(z)} e^{2\pi yk} e^{ i \left(x\arg(z) + y\ln|z|\right) }, \; k \in \Z
\]
Il y a deux cas qui déterminent le nombre de valeurs de $z^w$
\begin{enumerate}
    \item Si $y \neq 0$ alors $z^w$ prend une infinité de valeurs distinctes, puisque $e^{2\pi yk} \neq 1 \ \forall \ k \in \Z$
    \item Si $y = 0$ alors le nombre de valeurs dépend de $x$
          \begin{enumerate}
              \item Si $x \in \Q$, alors c'est le cas des racines entières [\ref{whole_roots}]
              \item Si $x \in \R \backslash \Q$, alors $z^w$ prend une infinité de valeurs avec une partie imaginaires qui varie et une partie réelle qui reste fixe.
          \end{enumerate}
\end{enumerate}
\begin{remark}
    Dans ce dernier cas, si $z, w \in \R$ alors, par convention, on prend $k = 0$ pour que $z^w \in \R$.
\end{remark}

\subsection{Théorème fondamental de l'algèbre}
Considérons le polynôme complexe $p(x)$ de degré $n \in \N$. Il est possible d'écrire $p(x)$ sous sa forme générale, qui est:
\[
    p(x) = a_n x^n + a_{n-1} x^{n-1} + \dots + a_1 x + a_0, \; a_j \in \C, \; a_n \neq 0
\]
\begin{definition}
    Les racines d'un polynôme complexe sont des valeurs complexes $t \in \C$ telles que $p(t) = 0$
\end{definition}
\begin{lemma}
    Il est possible d'écrire $p(x)$ en terme de ses $m$ racines, $\{t_1, t_2, \dots ,t_m\}$, c-à-d
    \[ p(x) = a_n(x - t_1)^{k_1} (x - t_2)^{k_2} \dots (x - t_m)^{k_m} \]
    où $k_j$ est la multiplicité de la racine $t_j$
\end{lemma}
\begin{lemma}
    \label{sum_multiplicity}
    La somme des $m$ multiplicités donne $n$, c-à-d $k_1 + k_2 + \dots + k_m = n$
\end{lemma}
D'après le lemme \ref{sum_multiplicity} nous pouvons formuler le théorème suivant:
\begin{theorem}[Théorème fondamental de l'algèbre]
    \label{theorem_fondamental_algebra}
    $p(x)$ a exactement $n$ racines complexes en comptant les multiplicités.
\end{theorem}
\unnumsec{Matrices}

\subsection{Définition d'une matrice}
\begin{definition}
    Soit $A$ une matrice de type $m$ par $n$. $A$ est aussi noté $A_{m \times n}$ et peut être représentée
    par un tableau qui contient $m$ lignes et $n$ colonnes ou $\left(A\right)_{kj}$ est l'élément
    à la ligne $k$ et colonne $j$. \[
        A = A_{m \times n} = \begin{pmatrix}
            a_{11} & a_{12}      & \dots  & a_{1n} \\
            a_{21} & a_{22}      & \dots  & a_{2n} \\
            \vdots &             & \ddots & \vdots \\
            a_{m1} & a_{m2}      & \dots  & a_{mn}
        \end{pmatrix}, \; \left(A\right)_{kj} \in F \ \forall \ k, j
    \]
\end{definition}
\begin{definition}
    L'ensemble des matrices de type $m \times n$ sur un corps $F$ est noté $M_{m \times n}(F)$ \\
    Si $F$ est un corps quelconque, alors $M_{m \times n}(F) = M_{m \times n}$
\end{definition}
\begin{note}
    Dans ce cours, on considère seulement le cas $F = \R$ ou $F = \C$
\end{note}
\begin{definition}
    Si $m = n$ alors $A_{m \times n} = A_{nn} = A_n$ est une matrice carrée.
\end{definition}
\begin{definition}
    La matrice composée uniquement de 0 pour un certain type $m \times n$ est écrite $\mathbb{O}$
\end{definition}
\begin{definition}
    La matrice identité de type $n$ est écrite $I_n = I$ avec $(I)_{kj} = \begin{cases}
            1 & \text{si} \ i = j    \\
            0 & \text{si} \ i \neq j
        \end{cases}$
\end{definition}
Soient $A \in M_n(F)$ et $a_{kj} \in F$ où $k, j \in \{0, 1, \dots, n\}$
\begin{definition}
    $A$ est triangulaire supérieure si $(A)_{kj} = \begin{cases}
            a_{kj} & \text{si} \ j \geq k \\
            0      & \text{si} \ j < k
        \end{cases}$
\end{definition}
\begin{definition}
    $A$ est triangulaire inférieure si $(A)_{kj} = \begin{cases}
            a_{kj} & \text{si} \ j \leq k \\
            0      & \text{si} \ j > k
        \end{cases}$
\end{definition}
\begin{remark}
    $A$ est triangulaire si $A$ est triangulaire supérieure ou inférieure
\end{remark}
\begin{definition}
    $A$ est diagonale si $(A)_{kj} = \begin{cases}
            a_{kj} & \text{si} \ i = j    \\
            0      & \text{si} \ i \neq j
        \end{cases}$
    \begin{remark}
        $I_n = \text{diag}\{ 1, 1, \dots, 1 \}$ ainsi que $\mathbb{O}_n = \text{diag}\{ 0, 0, \dots, 0 \}$
    \end{remark}
\end{definition}


\subsection{Opérations matricielles}

\subsubsection{Addition de matrices}
L'addition de deux matrices $A$ et $B$ existe seulement si $A$ et $B$ sont du même type.
L'addition est commutative, associative, possède un élément neutre et possède un inverse.
\[ (AB)_{kj} = a_{kj} + b_{kj} \]

\subsubsection{Multiplication par un scalaire}
La multiplication par un scalaire $\alpha$ existe pour une matrice $A$ de n'importe quel type. La multiplication
par un scalaire est commutative, associative, distributive et possède un inverse si $\alpha \neq 0$.
\[ (\alpha A)_{kj} = \alpha a_{kj}\]

\subsubsection{Produit matricielle}
Soient $A \in M_{m \times n}$ et $B \in M_{p \times q}$. 
Le produit matricielle $AB$ existe si $n = p$. Dans ce cas, la matrice résultante est de type $m \times q$ et l'élément à la position $k, j$ est calculé comme tel
\[ (AB)_{kj} = \sum_{s = 1}^{n} a_{ks} b_{sj} \]
Visuellement, on peut représenter la multiplication de $A$ et $B$ comme suit:

\begin{figure}[!htbp]
\begin{center}
\newcommand{\matMulFigUnit}{1 cm}
\tikzset{
    node style sp/.style={draw,circle,minimum size=\matMulFigUnit},
    node style ge/.style={circle,minimum size=\matMulFigUnit},
}
\begin{tikzpicture}[>=latex]

\matrix (A) [matrix of math nodes,
             nodes = {node style ge},
             left delimiter  = (,
             right delimiter = )] at (-6 * \matMulFigUnit,0)
{
  a_{11} & a_{12} & \ldots & a_{1n}  \\
  |[node style sp]| a_{21} & |[node style sp]| a_{22} & \ldots & |[node style sp]| a_{2n} \\
  \vdots & \vdots & \ddots & \vdots  \\
  a_{m1} & a_{m2} & \ldots & a_{mn}  \\
};
\node [draw,above=10pt] at (A.north)  { $A$ : \textcolor{red}{$m$ lignes} $n$ colonnes};

\matrix (B) [matrix of math nodes,
             nodes = {node style ge},
             left delimiter  = (,
             right delimiter = )] at (0,0)
{
  b_{11} & |[node style sp]| b_{12} & \ldots & b_{1q}  \\
  b_{21} & |[node style sp]| b_{22} & \ldots & b_{2q}  \\
  \vdots & \vdots & \ddots & \vdots  \\
  b_{n1} & |[node style sp]| b_{n2} & \ldots & b_{nq}  \\
};
\node [draw,above=10pt] at (B.north)  { $B$ : $n$ lignes \textcolor{red}{$q$ colonnes}};

\matrix (C) [matrix of math nodes,
             nodes = {node style ge},
             left delimiter  = (,
             right delimiter = )] at (6*\matMulFigUnit,0)
{
  c_{11} & c_{12} & \ldots & c_{1q} \\
  c_{21} & |[node style sp,blue]| c_{22} & \ldots & c_{2q} \\
  \vdots & \vdots & \ddots & \vdots \\
  c_{n1} & c_{n2} & \ldots & c_{nq} \\
};

\path (A)--(B) node[midway, black] {$\times$};
\path (B)--(C) node[midway, black] {$=$};

\draw[blue] (A-2-1.north) -- (A-2-4.north);
\draw[blue] (A-2-1.south) -- (A-2-4.south);
\draw[blue] (B-1-2.west)  -- (B-4-2.west);
\draw[blue] (B-1-2.east)  -- (B-4-2.east);

\node [draw,above=10pt] at (C.north) 
    {$ C = AB$ : \textcolor{red}{$m$ lignes}
                      \textcolor{red}{$q$ colonnes}};

\node [draw, below=10pt] at (B.south)
    {$(C)_{22} = (AB)_{22} = \displaystyle{\sum_{s = 1}^{n} a_{2s} b_{s2}} = a_{21}b_{12} + a_{22}b_{22} + \ldots + a_{2n}b_{n2}$
    };

\end{tikzpicture}
\end{center}
\caption{Représentation de la multiplication de deux matrices\protect\footnotemark[1]}
\end{figure}

\fancyfootnotetext{1}{Le figure vient de \href{https://texample.net/matrix-multiplication/}{https://texample.net/matrix-multiplication/}}

\subsubsection{Propriétés du produit matricielle}
Soient $A, B, C$ des matrices dont le produit matricielle entre eux existe. \\
Le produit matricielle est associatif, distributif et possède un élément neutre, soit $I$. \\
\underline{Important}: Le produit matricielle n'est, généralement, \underline{pas commutatif}
\[
    \begin{matrix}
        (AB)C = A(BC) & AB \neq BA & A(B + C) = AB + AC & (A + B)C = AC + BC & AI = IA = A
    \end{matrix}
\]

\subsubsection{Transposition}
\begin{definition}
    La transposée de $A \in M_{m \times n}$ est notée $A^T$ avec $\left(A^T\right)_{kj} = a_{jk}$.
    \begin{remark}
        Pour une matrice carrée, la transposée est une rotation des anti-diagonales par rapport à la grande diagonale.
    \end{remark}
\end{definition}

\subsubsection{Conjugé hermitien}
\begin{definition}
    Le conjugué hermitien de $A \in M_{m \times n}$ est noté $A^\dagger$ avec $A^\dagger = \left(A^T\right)^\star = \left(A^\star\right)^T$.
    Le conjugué hermitien est aussi appellé la transposée conjugué.
\end{definition}
\begin{remark}
    Pour $A \in M_{m \times n}(\R)$, $A^\dagger = A^T$
\end{remark}

\subsubsection{Propriétés de la transposée et du conjugué hermitien}
Soient $A, B \in M_n$, $\scalaire{\alpha}$. Alors on a les propriétés suivantes:

\begingroup
\renewcommand{\arraystretch}{1.5}

\begin{center}
    
    \begin{tabular}{c@{\hskip 1in} c}
        Involution & Multiplication par un scalaire \\
        $ \left( A^T \right)^T = A$ & $\left( \alpha A \right)^T = \alpha A^T$ \\ 
        $\left( A^\dagger \right)^\dagger = A $ & $\left( \alpha A \right)^\dagger = \alpha^\star A^\dagger$  \\[0.5em]
        Distribution sur l'addition   & Anti-distribution sur la multiplication \\ 
        $\left( A + B \right)^T = A^T + B^T$  & $\left( AB \right)^T = B^TA^T$ \\  
        $\left( A + B \right)^\dagger = A^\dagger + B^\dagger$ & $\left( AB \right)^\dagger = B^\dagger A^\dagger$
    \end{tabular}
\end{center}
\endgroup

\subsubsection{Symétrie, orthogonalité, hermitien et unitaire}
\begin{definition}
    Si $A^T = A$, alors $A$ est dite symétrique. Si $A^T = -A$, alors $A$ est dite anti-symétrique.
\end{definition}
\begin{definition}
    Si $A$ est une matrice carrée telle que $AA^T = A^TA = I$, alors $A$ est dite orthogonale.
\end{definition}
\begin{definition}
    Si $A^\dagger = A$, alors $A$ est dite hermitienne. Si $A^\dagger = -A$, alors $A$ est dite anti-hermitienne.
\end{definition}
\begin{definition}
    Si $A$ est une matrice carrée telle que $AA^\dagger = A^\dagger A = I$, alors $A$ est dite unitaire.
\end{definition}
\begin{remark}
    La symétrie et l'orthogonalité s'appliquent aux matrices réels.
    L'hermitien et l'unitaire s'appliquent aux matrices complexes.
\end{remark}

\subsubsection{Commutateur de matrice}
Soient $A, B \in M_n$, alors le commutateur de $A$ et $B$ est la matrice
\[ \left[ A, B \right] = AB - BA \]

\subsubsection{Propriétés du commutateur}
\begin{enumerate}
    \item $[A, B] = -[B, A]$
    \item $[\alpha A, B ] = [A, \alpha B] = \alpha [A, B]$
    \item $[A + B, C] = [A, C] + [B, C]$
    \item $[A, B]^T = -[A^T, B^T]$
    \item L'identité de Jacobi: $[[A, B], C] + [[C, A], B] + [[B, C], A] = \mathbb{O}$
\end{enumerate}

\subsubsection{Trace d'une matrice}
Soit $A \in M_n(F)$. Alors la trace de $A$ est un élément du coprs $F$ donné par
\[ \text{tr}A =  \sum_{j = 0}^{n} a_{jj} \]

\subsubsection{Propriétés de la trace}
\begin{enumerate}
    \item $\text{tr}\{ AB \} = \text{tr}\{ BA \}$
    \item $\text{tr}\{ A + B \} = \text{tr}A + \text{tr}B$
    \item $\text{tr}\{ \alpha A \} = \alpha \text{tr}A $
    \item $\text{tr}\{ A^T \} = \text{tr}A$, $\text{tr}\{ A^\dagger \} = \text{tr}A^\star$
    \item $\text{tr}\{ [A, B] \} = \text{tr}\{ AB - BA \} = \text{tr}\{ AB \} - \text{tr}\{BA\} = \text{tr}\{ AB \} - \text{tr}\{AB\} = 0$
\end{enumerate}

\subsection{Déterminant d'une matrice carrée}
\begin{definition}
    \label{definition_determinant}
    Le déterminant de $A \in M_{n}(F)$ noté $\det A$ ou $|A|$ est un élément dans le corps $F$ qui peut être caractérisé par trois propriétés:
\begin{enumerate}
    \item Le déterminant de la matrice identité est l'identité dans le corps $F$, c-à-d  \[ \det I = 1 \]
    \item Soit $C_j \in M_{n \times 1}$ une colonne de $A$. Le déterminant de $A$ doit satisfaire la propriété 
    \[ C_j = \alpha w + t \implies \det A = \alpha \det \begin{pmatrix}
                  C_1 & \ldots & w & \ldots & C_n
              \end{pmatrix} + \det \begin{pmatrix}
                  C_1 & \ldots & t & \ldots & C_n
              \end{pmatrix} \]
    \item Soient $C_j$ et $C_k$ deux colonnes de $A$ avec $k \neq j$. Le déterminant de $A$ doit satisfaire la propriété 
    \[ C_j = C_k \implies \det A = 0 \]
\end{enumerate}
\end{definition}
\begin{note}
    Les lignes de $A$ doivent aussi satisfaire les propriétés. 
\end{note}
Il existe plusieurs façons de calculer le déterminant. Ce document en couvre trois.

\subsubsection{Calcul du déterminant par récurrence}
Cette méthode de calculer le déterminant utilise les sous-matrices.
\begin{definition}
    \label{sous_matrice}
    Une sous-matrice d'indice $(k, j)$ de $A \in M_n$ est une matrice notée $M_{kj} \in M_{(n-1) \times (n-1)}$. On obtient $M_{kj}$ en biffant la k-ème ligne et la j-ème colonne de $A$. La sous-matrice est aussi appelée le mineur de $A$.
\end{definition}
Ainsi, on peut calculer le déterminant de $A$ en faisant l'expansion par la k-ème ligne.
\begin{theorem}
    Le calcul du déterminant de $A$ par récurrence sur $n$, la taille de la matrice, est 
    \[
        \det A = \begin{cases}
            a_{11},                                   & \textup{si } n = 1 \\
            \sum_{j = 1}^{n} (-1)^{k + j}a_{kj} \det{M_{kj}} & \textup{si } n > 1
        \end{cases}
    \]
\end{theorem}
\begin{remark}
    $k$ est l'indice d'une ligne de $A$, alors $k$ est un chiffre arbitraire avec la contrainte $1 \leq k \leq n$.
\end{remark}
\begin{corollary}
    Le calcul du déterminant de $A$ en faisant l'expansion par la j-ème colonne est
    \[
        \det A = \begin{cases}
            a_{11},                                   & \textup{si } n = 1 \\
            \sum_{k = 1}^{n} (-1)^{k + j}a_{kj} \det{M_{kj}} & \textup{si } n > 1
        \end{cases}
    \]
\end{corollary}
\begin{remark}
    La seule différence entre l'expansion par une ligne ou une colonne est l'indice de la somme. L'indice est $j$ si on développe par la k-ème ligne, sinon l'indice est $k$ si on développe par la j-ème colonne.
\end{remark}

\subsubsection{Calcul du déterminant par les permutations}
Commençons par définir quelques concepts qui seront utilisés dans cette méthode.
\begin{definition}
    Une permutation des nombres $1, 2, \dots, n$ est un arrangement de ces $n$ nombres sans répétition. Notons $\sigma = \begin{pmatrix}
        \sigma_1 & \sigma_2 & \dots & \sigma_n 
    \end{pmatrix}$ une permutation des $n$ nombres o\`u $\sigma_j \in \{1, \dots, n \}$ et \\ $\sigma_j \neq \sigma_k \iff j \neq k$
\end{definition}
\begin{definition}
    Un couple d'élément $\sigma_j, \sigma_k \in \sigma$ est dit dans le désordre si $j < k$ mais $\sigma_j > \sigma_k$
\end{definition}
\begin{definition}
    Le nombre de désordres d'une permutation $\sigma$ est le nombre de couples de $\sigma$ dans le désordre
\end{definition}
\begin{definition}
    La signature de $\sigma$ est $\text{sign}(\sigma) = (-1)^{\text{nombre de désordre de $\sigma$}}$ 
\end{definition}
\begin{definition}
    $S_n$ est l'ensemble qui contient toutes les permutations des nombres $1, 2, \dots, n$
\end{definition}

\begin{theorem}
    Le calcul du déterminant de $A$ par les permutations est 
    \[
        \det A = \sum_{\sigma \in S_n} \textup{sign}(\sigma) \prod_{i = 1}^{n}a_{j \sigma_j}
    \]
\end{theorem}

\subsubsection{Calcul du déterminant par les propriétés caractéristiques}
Il est possible de prouver des nouvelles propriétés à partir des propriétés caractéristiques du déterminant [\ref{definition_determinant}]. Ces nouvelles propriétés ajoutées aux propriétés caractéristiques nous donnent cinq propriétés qui sont utiles pour calculer le déterminant.
\begin{enumerate}
    \item Échanger deux colonnes ou lignes change le signe du déterminant
          \[\det \arraycolsep=0.3\arraycolsep \begin{pmatrix}
                  C_1 & \ldots & C_j & \ldots & C_k & \ldots & C_n
              \end{pmatrix} = -\det \begin{pmatrix}
                  C_1 & \ldots & C_k & \ldots & C_j & \ldots & C_n
              \end{pmatrix}\]
    \item Si deux colonnes ou lignes sont pareilles, alors le déterminant est nul
          \[\det \arraycolsep=0.3\arraycolsep \begin{pmatrix}
                  C_1 & \ldots & C_j & \ldots & C_j & \ldots & C_n
              \end{pmatrix} = 0\]
    \item Multiplier une colonne ou ligne par un scalaire revient à multiplier le déterminant par ce même scalaire
          \[\det \arraycolsep=0.3\arraycolsep \begin{pmatrix}
                  C_1 & \ldots & \alpha C_j & \ldots & C_n
              \end{pmatrix} = \alpha \det \begin{pmatrix}
                  C_1 & \ldots & C_j & \ldots & C_n
              \end{pmatrix}\]
    \item Ajouter une colonne ou ligne multipliée par un scalaire à une autre colonne ou ligne
          ne change pas le déterminant
          \[ \det \arraycolsep=0.3\arraycolsep \begin{pmatrix}
                  C_1 & \ldots & C_j & \ldots & C_k & \ldots & C_n
              \end{pmatrix} = \det \begin{pmatrix}
                  C_1 & \ldots & C_j + \alpha C_k & \ldots & C_k & \ldots & C_n
              \end{pmatrix}\]
    \item Si on peut décomposer une colonne ou ligne en la somme de deux colonnes ou lignes,
          alors le déterminant est la somme des déterminants des deux matrices qui ont chacune
          une décomposition de la colonne ou ligne
          \[ \det \arraycolsep=0.3\arraycolsep \begin{pmatrix}
                  C_1 & \ldots M + N & \ldots & C_n
              \end{pmatrix} = \det \begin{pmatrix}
                  C_1 & \ldots M & \ldots & C_n
              \end{pmatrix} + \det \begin{pmatrix}
                  C_1 & \ldots N & \ldots & C_n
              \end{pmatrix} \]
\end{enumerate}
Ces cinq propriétés nous permettent de transformer $\det A$ en la forme $\alpha \det I$. Par définition, 
$\det I = 1$, alors nous avons calculé le déterminant de $A$

\subsubsection{Propriétés du déterminant}
Soient $A, B \in M_n, \ \alpha$ scalaire, alors \begin{enumerate}
    \item $\det A^T = \det A$
    \item $\det(\alpha A) = \alpha^n \det(A)$
    \item $\det (A^\star) = (\det A)^\star$
    \item $\det A^\dagger = (\det A)^\star$
    \item $\det(AB) = \det(A) \det(B)$
    \item $\det(A^m) = (\det A)^m$
    \item Généralement, $\det(A + B) \neq \det A + \det B$
    \item $\det(\text{diag}\{a_1, \ a_2, \ldots, a_n \}) = a_1 a_2 \dots a_n$
    \item Si $A$ est triangulaire alors $\det A = a_{11} a_{22} \dots a_{nn}$
\end{enumerate}

\subsection{Matrices inverses}
\begin{definition}
    $A \in M_{n \times n}$ est inversible $\iff \exists \ B$ t.q. $AB = BA = I$. \\
    Alors $B$ est noté $A^{-1}$ et est appellée la matrice inverse de $A$
\end{definition}
\subsubsection{Propriétés de l'inverse}
Soit $A \in M_{n \times n}(\C)$ inversible \begin{enumerate}
    \item $AB = \mathbb{O} \implies B = \mathbb{O}$
    \item Si $AC = BA = I \implies B = BI = B(AC) = (BA)C = IC = C $, l'inverse à gauche et à droite sont égales
    \item $(A^{-1})^{-1} = A$, \quad $(A^T)^{-1} = (A^{-1})^{T}$, \quad $(A^\dagger)^{-1} = (A^{-1})^{\dagger}$
    \item $(AB)^{-1} = B^{-1}A^{-1}$
\end{enumerate}

\subsubsection{Calculer la matrice inverse avec la matrice adjointe}
Commençons par définir les termes qui seront utilisés dans le calcul de l'inverse de $A$
\begin{definition}
     Le cofacteur d'indice $(k, j)$ de $A$, noté $c_{kj}$, est donné par
    \[ c_{kj} = (-1)^{k + j}\det M_{kj} \]    
    où $M_{kj}$ étant la sous-matrice d'indice $(k, j)$ de $A$ de la définition \ref{sous_matrice}
\end{definition}
\begin{definition}
     La matrice adjointe de $A$, notée $\text{adj} A$, est donnée par
\[ \left(\text{adj} A\right)_{kj} = \left( c_{jk} \right) \]
\end{definition}
\begin{theorem}
    Le calcul de l'inverse de $A$ avec sa matrice ajointe et son déterminant est
    \[ A^{-1} = \frac{1}{\det A} \ \textup{adj}A \]
\end{theorem}
Cette manière de calculer la matrice inverse nous permet de formuler ce théorème:
\begin{theorem}
    \label{inversible_small}
    $A$ est inversible $\iff \det A \neq 0$
\end{theorem}

\subsection{Systèmes d'équations linéaires}
\begin{definition}
    Une équation linéaire sur un corps $F$ est une équation de la forme 
    \[ a_1x_1 + a_2x_2 + \ldots + a_nx_n = b, \ a_j, b \in F \]
    On appelle $a_j$ les coefficients, $b$ le terme constant et $x_j$ les inconnus.
\end{definition}
\begin{definition}
    Un système de $m$ d'équations linéaires est un ensemble d'équations linéaires, noté:
    \[ \begin{cases}
            a_{11}x_1 + a_{12}x_2 + \ldots + a_{1n}x_n & = b_1 \\
            a_{21}x_1 + a_{22}x_2 + \ldots + a_{2n}x_n & = b_2 \\
            \vdots                                             \\
            a_{m1}x_1 + a_{m2}x_2 + \ldots + a_{mn}x_n & = b_m
        \end{cases} \]
\end{definition}
\begin{definition}
    Un système linéaire est dit compatible s'il admet au moins une solution et incompatible sinon. Une solution est une matrice de taille $n \times 1$
\end{definition}
\subsubsection{Représentation matricielle d'un système d'équation linéaire}
\begin{theorem}
    Chaque système de $m$ équations linéaires contenant $n$ inconnus peut être représenté par une équation matricielle de la forme $AX = B$ où $A \in M_{m \times n}, \ X \in M_{n \times 1}, \ B \in M_{m \times 1}$, 
    \[
        \begin{cases}
            a_{11}x_1 + a_{12}x_2 + \ldots + a_{1n}x_n & = b_1 \\
            a_{21}x_1 + a_{22}x_2 + \ldots + a_{2n}x_n & = b_2 \\
            \vdots                                             \\
            a_{m1}x_1 + a_{m2}x_2 + \ldots + a_{mn}x_n & = b_m
        \end{cases} \iff \begin{pmatrix}
            a_{11} & a_{12}      & \dots  & a_{1n} \\
            a_{21} & a_{22}      & \dots  & a_{2n} \\
            \vdots & \phantom{a} & \ddots & \vdots \\
            a_{m1} & a_{m2}      & \dots  & a_{mn}
        \end{pmatrix} \begin{pmatrix}
            x_1    \\
            x_2    \\
            \vdots \\
            x_n
        \end{pmatrix} = \begin{pmatrix}
            b_1    \\
            b_2    \\
            \vdots \\
            b_m
        \end{pmatrix}
    \]
\end{theorem}
\begin{definition}
    Dans la forme matricielle d'un système d'équation linéaire, $AX = B$, $A$ est dite la matrice de coefficient, $X$ la matrice d'inconnus et $B$ la matrice de terme constant.
\end{definition}
\begin{definition}
    Dans la forme $AX = B$ d'un système d'équation linéaire, on peut omettre la matrice d'inconnue puisqu'elle est toujours la même dans tous les systèmes linéaires. L'équation matricielle sans la matrice d'inconnue, dite la matrice augmentée du système, est notée $(A|B)$
\end{definition}

\subsubsection{Forme échelonnée d'un système}
\begin{definition}
    Une matrice est de la forme échelonnée si \begin{enumerate}
    \item La ligne qui précède une ligne non-nulle est non-nulle
    \item Le premier coefficient non-nul d'une ligne non-nulle, appelé le pivot, est plus à
          gauche que le pivot de la ligne suivante.
    \end{enumerate}
\end{definition}
La forme échelonnée de la matrice augmentée d'un système linéaire nous donnent un système beaucoup plus simple à résoudre. Pour obtenir la forme échelonnée d'une matrice,
il suffit d'appliquer les opérations élémentaires sur ces lignes. Ces opérations sont \begin{enumerate}
    \item Échanger deux lignes, $L_j \leftrightarrow L_k$
    \item Additionner une ligne à une autre qui est multipliée par un scalaire $L_j \mapsto L_j + \alpha L_k$
    \item Multiplier une ligne par un scalaire non-nul, $L_j \mapsto \alpha L_j, \ \alpha \neq 0$
\end{enumerate}

\subsubsection{Rang d'une matrice}
\begin{definition}
    Le rang de $A$, noté $\text{rg}A$, est le nombre de lignes non-nulles de la forme échelonnée de $A$.
\end{definition}
\begin{lemma}
    On peut utiliser le rang de la matrice échelonnée d'un système à $n$ inconnus pour déterminer si ce système est compatible ou incompatible.
    \begin{enumerate}
        \item Si $\textup{rg}(A) < \textup{rg}(A|B)$, alors le système n'a pas de solution, il est incompatible.
        \item Si $\textup{rg}(A) = \textup{rg}(A|B) = n$, alors le système a une unique solution, il est compatible.
        \item Si $\textup{rg}(A) = \textup{rg}(A|B) < n$, alors le système a une infinité de solutions, il est compatible.
    \end{enumerate}
\end{lemma}
\begin{remark}
    Si $A^\prime$ est la forme échelonnée de $A$, alors $\det A = \alpha \det A^\prime, \ \alpha 
 \ \text{scalaire}$.
    On peut donc étendre le théorème \ref{inversible_small} avec le rang.
\end{remark}
\begin{theorem}
    \label{theorem_extended_once}
    $\det A \neq 0 \iff A $ est inversible $\iff \textup{rg}(A) = n \iff AX=B $ a une unique solution
\end{theorem}
\begin{remark}
    L'unique solution dans ce cas est $X = A^{-1}B$, puisque $A$ est inversible.
\end{remark}

\subsubsection{Calculer l'inverse d'une matrice}
Il est possible de calculer l'inverse de $A$ en échelonnant le système $(A|I)$. Le
système échelonné va donner $(A|I) \sim (I|B)$, avec $B = A^{-1}$.

\subsubsection{Système homogène}
\begin{definition}
    Un système homogène est un système de la forme $AX = \mathbb{O}$
\end{definition}
\begin{remark}
    Un tel système est toujours compatible avec la solution triviale $X = \mathbb{O}$
\end{remark}
\begin{theorem}
    Soit le système homogène $AX = \mathbb{O}$. Si la matrice $A$ est inversible, alors le système a seulement la solution triviale, sinon il contient une infinité de solutions.
\end{theorem}

\subsubsection{Noyau d'un système}
\begin{definition}
    Le noyau de $A \in M_{m \times n}$, noté $\text{N}(A)$, est l'ensemble de toutes les solutions du système homogène
    $AX = \mathbb{O}$, soit 
    \[ \text{N}(A) = \left\{ X \in M_{n \times 1} \ | \ AX = \mathbb{O}  \right\} \]
\end{definition}
On peut reformuler le théorème \ref{theorem_extended_once} avec les systèmes homogènes
\begin{theorem}
    \label{inversible_rank_kernel_thm}
    $\det A \neq 0 \iff A $ est inversible $\iff \textup{rg}(A) = n \iff \textup{N}(A) = \mathbb{O} $
\end{theorem}
\unnumsec{Espace vectoriel de dimension finie}
\begin{definition}
      Un ensemble $V$ est un espace vectoriel sur un corps $F$
      si \begin{enumerate}[1)]
            \item $V$ est fermé sous l'addition, c-à-d 
            \[ \forall \ (v_1, v_2 \in V), \ v_1 + v_2 \in V \]
            \item $V$ est fermé sous la multiplication par un scalaire, c-à-d, \[ \forall \ (v \in V, \ \alpha \in F), \ \alpha v \in V \]
      \end{enumerate}
      Les éléments de $V$ sont appelés \guillemetleft \ vecteurs \guillemetright.
\end{definition}
\begin{remark}
      L'espace vectoriel le plus simple est $V = \{ 0_v \}$, où $0_v$ est l'élément nul.
\end{remark}
\begin{definition}
      Un scalaire $\alpha$ est un élément du corps $F$ associé à l'espace vectoriel.
\end{definition}
\begin{note}
    Dans notre cas, ce corps est soit les nombres réels $\R$ ou les nombres complexes $\C$
\end{note}

\subsection{Base d'un espace vectoriel}
Une base d'un espace vectoriel est une manière d'encoder l'information d'un vecteur. Commençons par définir les concepts utilisés dans la définition d'une base. Cette sous-section va utiliser ces deux objets, soit $V$ un espace vectoriel et $S = \{ u_1, u_2, \ldots, u_m \} \subset V$

\subsubsection{Combinaison linéaire}
\begin{definition} Une combinaison linéaire de $S$ est la somme des éléments $S$ multipliés par des scalaires $\alpha_1, \alpha_2, \ldots, \alpha_m$, c.-à-d. 
\[ \sum_{j = 1}^{n}\alpha_j u_j = \alpha_1 u_1 + \alpha_2 u_2 + \ldots + \alpha_m u_m \]
\end{definition}

\subsubsection{Ensemble générateur}
\begin{definition}
      $S$ est un ensemble générateur de $V$ si on peut écrire tout vecteur de $V$ comme une combinaison linéaire de $S$, c.-à.-d.
      \[ \forall v \in V  \ \exists \alpha_1, \ldots, \alpha_m \text{ scalaire t.q. } v = \alpha_1 u_1 + \ldots + \alpha_m u_m\]
\end{definition}

\subsubsection{Indépendance linéaire}
\begin{definition}
      $S$ est linéairement indépendant si la combinaison linéaire de $S$ donne 0 seulement quand tous les scalaires sont 0, c.-à.-d.
      \[
            \alpha_1 u_1 + \alpha_2 u_2 + \ldots + \alpha_n u_n = 0_v \implies \alpha_1 = \alpha_2 = \ldots = \alpha_n = 0
      \]
      sinon $S$ est linéairement dépendant ou lié.
\end{definition}

\subsubsection{Déterminer l'indépendance linéaire d'un ensemble de vecteurs colonne}
Supposons que $u_j \in S$ est un vecteur colonne de taille $n \times 1$. Notons k-ème élément du vecteur $u_j$ comme $u_{jk}$. Trouvons une condition pour l'indépendance linéaire de $S$
\begin{align*}
    & \alpha_1 u_1 + \alpha_2 u_2 + \ldots + \alpha_n u_n = \mathbb{O} \\[0.5em]
    \iff & \alpha_1 \begin{pmatrix}
                  u_{11} \\
                  u_{12} \\
                  \vdots   \\
                  u_{1n}
            \end{pmatrix} + \alpha_2 \begin{pmatrix}
                  u_{21} \\
                  u_{22} \\
                  \vdots   \\
                  u_{2n}
            \end{pmatrix} + \dots + \alpha_n \begin{pmatrix}
                  u_{n1} \\
                  u_{n2} \\
                  \vdots   \\
                  u_{nn}
            \end{pmatrix} = \mathbb{O} \\[0.5em]
    \iff & \begin{pmatrix}
        \alpha_1 u_{11} + \alpha_2 u_{12} + \ldots + \alpha_n u_{1n} \\
        \alpha_1 u_{21} + \alpha_2 u_{22} + \ldots + \alpha_n u_{2n} \\
        \vdots \\
        \alpha_1 u_{n1} + \alpha_2 u_{n2} + \ldots + \alpha_n u_{nn} \\
    \end{pmatrix} = \mathbb{O} \\[0.5em]
     \iff & \begin{pmatrix}
        u_{11} & u_{12} & \ldots & u_{1n} \\
        u_{21} & u_{22} & \ldots & u_{2n} \\
        \vdots & & \ddots & \vdots\\
        u_{n1} & u_{n2} & \ldots & u_{nn} \\
    \end{pmatrix} \begin{pmatrix}
        \alpha_1 \\
        \alpha_2 \\
        \vdots \\
        \alpha_n
    \end{pmatrix} = \mathbb{O}
\end{align*}
Posons $M$ comme la matrice de coefficient du système. Nous pouvons exprimer la matrice augmentée du système comme tel: $(M | \mathbb{O})$. Nous pouvons utiliser le théorème \ref{inversible_rank_kernel_thm} pour arriver à deux conclusions: 
\begin{enumerate}[1.]
      \item $\text{rg}(M) < m \implies S$ est lié.
      \item $\text{rg}(M^T) = m \implies S$ est linéairement indépendant.
\end{enumerate}
% Add the other way of doing M^T to make it easier
Note: Échelonner $M^T$ donne des résultats plus simples à interpréter puisque
$0 \leq \text{rg}(M^T) \leq m$. Si la forme échelonnée de $M^T$ contient une ligne nulle,
on sait automatiquement que $S$ est lié, ce qui n'est pas le cas si la forme échelonnée de $M$ 
contient une ligne nulle.

\subsubsection{Base d'un espace vectoriel}
\begin{definition}
      Un ensemble $B = \{ u_1, u_2, \ldots, u_n \} \subset V$, est une base de $V$ si
      $B$ est un ensemble générateur de $V$ et $B$ est linéairement indépendant.
\end{definition}
\begin{definition}
      La dimension de $V$ est $\dim_F V = |B|$, ou $|B|$ est le nombre d'éléments dans la base $B$
      de $V$ et $F$ est le corps de l'espace vectoriel.
\end{definition}
\begin{theorem}
      Le nombre de vecteurs dans une base de $V$ ne dépend pas de la base choisie.
\end{theorem}
\begin{remark}
      Le théorème qui précède nous permet de définir la dimension de $V$ comme étant le
      nombre de vecteurs dans une base de $V$, puisque toutes les bases de $V$ contiennent
      le même nombre de vecteurs.
\end{remark}
\paragraph{Base canonique:}
Un espace vectoriel a souvent une base canonique, soit une base plus naturelle à utiliser.
Par exemple, la base canonique de $\R^n$ est $S = \{ e_1, e_2, \ldots, e_n \}$.
On peut donc dire que $\dim_\R \R^n = n$

\subsubsection{Représentation d'un vecteur dans une base}
Soit $V$ un espace vectoriel avec $\dim_F V = n$, et $B = \{ u_1, u_2, \ldots, u_n \}$ une base de $V$. \\
Cela veut dire que $ \forall \ v \in V  \ \exists \ \alpha_1, \ldots, \alpha_n \text{ scalaire t.q. } v = \alpha_1 u_1 + \ldots + \alpha_n u_n$
\begin{definition}
      Les scalaires $\alpha_1, \ldots, \alpha_n$ sont appellés les coordonnées de $v$ dans la base $B$.
\end{definition}
\begin{definition}
      La représentation de $v$ dans la base $B$ est notée $[v]_B$ et est exprimée
      \[
            [v]_B = \begin{pmatrix}
                  \alpha_1 \\
                  \alpha_2 \\
                  \vdots   \\
                  \alpha_n
            \end{pmatrix} \in F^n
      \]
\end{definition}

\subsubsection{Propriétés d'une base d'un espace vectoriel}
\begin{enumerate}[a)]
      \item Soit $v_1, v_2 \in V$, alors $v_1 = v_2 \iff [v_1]_B = [v_2]_B$
      \item La représentation dans une base est linéaire, ce qui veut dire qu'elle satisfait ces deux propriétés:
            \begin{enumerate}[1.]
                  \item $[v_1 + v_2]_B = [v_1]_B + [v_2]_B$
                  \item $[\alpha v]_B = \alpha [v]_B$
            \end{enumerate}
      \item Soit $S = \{ u_1, u_2, \ldots, u_n \} \subset V$ \\
            $S^\prime = \{ [u_1]_B, [u_2]_B, \ldots, [u_n]_B \}$ linéairement indépendant $\iff S$ linéairement indépendant
\end{enumerate}

\subsection{Matrice de changement de base}

Soit $V$ un espace vectoriel avec $\dim_F V = n$ \\
Soit $B = \left\{ u_1, u_2, \dots, u_n \right\}$ et $B^\prime = \left\{ u_1^\prime, u_2^\prime, \dots, u_n^\prime \right\}$ deux bases de $V$ 
\begin{definition}
      La matrice de changement (ou de passage) de la base $B^\prime$ à $B$ est notée
      $P^{B^\prime}_B$ et est donnée par $P^{B^\prime}_B = \begin{bmatrix}
           [ u_1^\prime]_B & [u_2^\prime]_B & \dots & [u_n^\prime]_B
      \end{bmatrix}$. Les bases sont liées par la matrice de passage par l'équation 
      \[P^{B^\prime}_B [v]_{B^\prime} = [v]_B\]
\end{definition}
Note: Il a plusieurs notations pour la matrice de passage. La notation utilisée
dans cette définition est équivalente à $\underset{B^\prime \to B}{P}$
\begin{lemma}
      La matrice de passage de $B$ à $B^\prime$ est l'inverse de la matrice de passage de 
      $B^\prime$ à $B$, soit \[P^B_{B^\prime} = \left(P^{B^\prime}_B\right)^{-1}\]
\end{lemma}
\begin{lemma}
      La matrice de passage est unique.
\end{lemma}

\subsection{L'indépendance linéaire des fonctions} % Retraviller le formatting
Soit $V$ l'espace vectoriel de toutes les fonctions $f: \R \to \R$ avec des scalaire réels \\
La dimensions de $V$ est infinie et il n'y a pas de base dans $V$ \\
Soit ${v_1, v_2, \dots, v_n} \subset V$ et $S = \text{span}\{v_1, v_2, \dots, v_n\}$. \\
Pour déterminer l'indépendance linéaire de $S$, considérons $F(x) = \alpha_1 v_1 + \alpha_2 v_2 + \dots + \alpha_n v_n$ \\
Supposons que $S$ est linéairement indépendant, c-à-d que $F(x) = 0 \ \forall \ x \in \R$ \\
Il faut ensuite prendre $x_1, x_2, \dots, x_n$ valeurs et évaluer $F(x_1), F(x_2), \dots, F(x_n)$. \\
Le résultat va être $n$ équations de la forme $\alpha_1 v_1(x_j)+ \alpha_2 v_2(x_j)+ \dots + \alpha_n v_n(x_j) = 0$ \\
\underline{Important:} Pour que la méthode fonctionne, il faut que les $n$ équations soient uniques. 
En d'autres termes, si $F(a) = F(b)$ alors ni $a$ ni $b$ ne peuvent faire partie des $n$ valeurs choisies. \\
Maintenant, il faut résoudre le système homogène $\begin{cases}
      \alpha_1 v_1(x_1)+ \alpha_2 v_2(x_1)+ \dots + \alpha_n v_n(x_1) = 0 \\
      \alpha_1 v_1(x_2)+ \alpha_2 v_2(x_2)+ \dots + \alpha_n v_n(x_2) = 0 \\
      \vdots \\
      \alpha_1 v_1(x_n)+ \alpha_2 v_2(x_n)+ \dots + \alpha_n v_n(x_n) = 0 \\
\end{cases}$ \\
La résolution du système va aboutir à deux possibilités: \begin{enumerate}
      \item Le système homogène a seulement la solution nulle $\implies S$ est linéairement indépendant
      \item Le système homogène a une infinité de solutions. Dans ce cas, on doit construire
      une nouvelle fonction $\hat{F}(x) = \beta_1 v_1 + \beta_2 v_2 + \dots + \beta_n v_n$ où $\beta_1, \beta_2, \dots, \beta_n$
      est une solution non-nulle du système. Il faut alors démontrer que $\hat{F}(x) = 0 \ \forall \ x \in \R$. 
      Si c'est le cas, alors on peut conclure que $S$ est lié.
\end{enumerate}

\subsection{Espaces vectoriels munis d'un produit scalaire}

\subsubsection{Produit scalaire réel}
Soit $V$ un espace vectoriel réel
\begin{definition}
      \label{scpr_Real}
      Un produit scalaire dans $V$ est une application $\myfunc{\scpr{\cdot}{\cdot}}{V \times V}{\R}{u, v \in V}{\scpr{u}{v}}$, t.q.
      \begin{align*}
            &1. \quad \scpr{u}{v} = \scpr{v}{u}& &2. \quad \scpr{u}{v + \alpha w} = \scpr{u}{v} + \alpha \scpr{u}{w}& &3. \quad \scpr{u}{u} \geq 0 \ \text{et} \ \scpr{u}{u} = 0 \iff u = 0&
      \end{align*}
\end{definition}
\begin{remark}
      Le produit scalaire réel n'est pas unique. 
\end{remark}
\begin{definition}
      Une matrice $A \in M_{n \times n}(\R)$ est définie positive si $x^T A x > 0 \ \forall \ x \neq 0, \ x \in R^n$
\end{definition}
\begin{lemma}
      Soit $x,y \in \R^n$ et $A \in M_{n \times n}(\R)$ symétrique et définie positive, alors tout produit scalaire dans $\R^n$ 
      peut être écrit comme tel, $\scpr{x}{y} = x^T A y$. On note ce produit scalaire modifié $\scpr{u}{v}_A$
\end{lemma}
\begin{definition}
      Le cas où $A = I \implies \scpr{x}{y} = x^T I y = x^T y$ est appellé le produit scalaire 
      canonique (ou euclédien) dans $\R^n$
\end{definition}

\subsubsection{Produit scalaire complexe}
Soit $V$ un espace vectoriel complexe. Le produit scalaire complexe est défini de manière
similaire au produit scalaire réel [\ref{scpr_Real}] à l'exception d'une restriction de plus.
\begin{definition}
      Un produit scalaire dans $V$ est une application $\myfunc{\scpr{\cdot}{\cdot}}{V \times V}{\C}{u, v \in V}{\scpr{u}{v}}$, t.q.
      \begin{align*}
            &1. \quad \scpr{u}{v} = \scpr{v}{u}^*& &2. \quad \scpr{u}{v + \alpha w} = \scpr{u}{v} + \alpha \scpr{u}{w}& &3. \quad \scpr{u}{u} \geq 0 \ \text{et} \ \scpr{u}{u} = 0 \iff u = 0&
      \end{align*}
\end{definition}
\begin{remark}
      La définition implique que $\scpr{\alpha v + \beta u}{w} = \alpha^* \scpr{v}{w} + \beta^* \scpr{u}{w}$
\end{remark}
\begin{lemma}
      Soit $x,y \in \C^n$ et $A \in M_{n \times n}(\C)$ hermitienne et définie positive, alors tout produit scalaire dans $\C^n$ 
      peut être écrit comme tel, $\scpr{x}{y}_A = x^\dagger A y$.
\end{lemma}
\begin{definition}
      Le produit scalaire canonique dans $\C^n$ est $\scpr{x}{y} = x^\dagger y$
\end{definition}

\subsubsection{Orthogonalité entre deux vecteurs}
Soit $V$ un espace vectoriel muni d'un produit scalaire
\begin{definition}
      $x, y \in V$ sont orthogonaux $\iff \scpr{x}{y} = 0$. Alors on note $x \perp y$.
\end{definition}
\begin{remark}
      L'orthogonalité entre deux vecteurs est toujours par rapport au produit scalaire utilisé
\end{remark}

\subsubsection{Norme d'un vecteur}
Soit $V$ un espace vectoriel muni d'un produit scalaire
\begin{definition}
      La norme de $x \in V$ par rapport au produit scalaire dans $V$ est $\norm{x} = \sqrt{\scpr{x}{x}}$
\end{definition}
\begin{remark}
      La norme d'un vecteur change par rapport au produit scalaire utilisé
\end{remark}
\begin{definition}
      Un vecteur de norme 1 est dit un vecteur unitaire.
\end{definition}
\begin{lemma}
      Tout vecteur $x$ peut être transformé en vecteur unitaire avec la formule $\hat{x} = \frac{x}{\norm{x}}$
\end{lemma}

\subsubsection{Distance entre deux vecteurs}
Soit $V$ un espace vectoriel muni d'un produit scalaire
\begin{definition}
      La distance entre $x, y \in V$ est $\text{dist}(x, y) = \norm{x - y}$
\end{definition}
\begin{remark}
      Encore une fois, la distance entre deux vecteurs change par rapport au produit scalaire utilisé
\end{remark}

\subsubsection{Inégalité du produit scalaire}
Soit $V$ un espace vectoriel muni d'un produit scalaire, $u, v \in V$
\begin{theorem}
      L'inégalité de Cauchy-Schwarz est $\left|\scpr{u}{v}\right| \leq \norm{u}\norm{v}$
\end{theorem}
\begin{theorem}
      L'inégalité du triangle est $\norm{u + v} \leq \norm{u} + \norm{v}$
\end{theorem}

\subsection{Bases orthonormales}
Soit $V$ un espace vectoriel muni d'un produit scalaire et $S = \{u_1, u_2, \dots, u_m\} \subset V$
\begin{definition}
      $S$ orthonormal $\iff$
      $u_{k_1} \perp u_{k_2} \ \forall \ k_1 \neq k_2$ et $\norm{u_k} = 1 \ \forall \ k = 1, 2, \dots, m$ 
\end{definition}
\begin{definition}
      Le delta de Kronecker est $\delta_{kj} = \begin{cases}
            0 & \text{si} \ k \neq j \\
            1 & \text{si} \ k = j \\
      \end{cases}$
\end{definition}
\begin{remark}
      Le delta de Kronecker pour $i, j \in \{0, 1, \dots, n\}$ est la matrice identité $n \times n$.
      De plus, on peut utiliser le delta de Kronecker pour simplifier la définition d'un ensemble orthonormé.
\end{remark}
\begin{definition}
      $S$ orthonormal $\iff \scpr{u_k}{u_j} = \delta_{kj} \ \forall \ u_k, u_j \in S$ 
\end{definition}
\begin{definition}
      $B$ est une base orthonormale (ou orthonormée) dans $V$ si $B$ est une base dans $V$ et 
      $B$ est un ensemble orthonormal.
\end{definition}
Soit $B$ une base orthonormale de $V$. Notons le produit scalaire dans $V$ comme $\scpr{\cdot}{\cdot}_V$
\begin{theorem}
      Pour $x, y \in V$, on a $\scpr{x}{y}_V = [x]_B^\dagger [y]_B = \scpr{[x]_B}{[y]_B}_{\C^n}$
\end{theorem}
\begin{remark}
      Il est possible de traduire tout produit scalaire dans $V$ au produit scalaire canonique de
      $\C^n$ à l'aide d'une base orthonormale de $V$.
\end{remark}

\subsection{Projection orthogonale}
Soit $R = \{u_1, u_2, \dots, u_m\} \subset V$ un ensemble orthonormal avec $\dim V = n > m$ \\
Considérons le sous-espace $W = \text{span}\{R\} \subset V$ qui a $R$ comme base orthonormale
\begin{definition}
      Soit $v \in V$, alors $S_W(v) = \sum\limits_{k = 1}^{m}\scpr{u_k}{v}{u_k}$
\end{definition}
\begin{theorem}
      Le vecteur $S_W(v) \in W$ et $S_W(v) \perp w$ $\forall \ v \in V, \ w \in W$
\end{theorem}

\subsection{Orthonormalisation de Gram-Schmidt}
Soit $W = \{w_1, w_2, \dots, w_m\}$ un ensemble linéairement indépendant. L'orthonormalisation 
de Gram-Schmidt va produire un ensemble orthonormal $S = \{u_1, u_2, \dots, u_m\}$ à partir
des vecteurs de $W$. 
\begin{align*}
      &\text{Étape} \ 1. \quad \text{Posons} \ u_1 = \frac{w_1}{\norm{w_1}}& &\text{Alors} \ u_1 \ \text{est unitaire} \\
      &\text{Étape} \ 2. \quad \text{Posons} \ v_2 = w_2 - S_{<u_1>}(w_2)& &\text{Alors} \ v_2 \perp u_1 \\
      &\text{Étape} \ 3. \quad \text{Posons} \ u_2 = \frac{v_2}{\norm{v_2}}& &\text{Alors} \ u_2 \ \text{est unitaire et} \ u_2 \perp u_1 \\
      \vdots \\
      &\text{Étape} \ 2m - 2. \quad \text{Posons} \ v_{m} = w_{m} - S_{<u_1, \dots, u_{m - 1}>}(w_{m})& &\text{Alors} \ v_{m} \perp \{u_1, \dots, u_{m - 1}\} \\
      &\text{Étape} \ 2m - 1. \quad \text{Posons} \ u_{m} = \frac{v_{m}}{\norm{v_{m}}} & &\text{Alors} \ u_{m} \ \text{est unitaire et} \ u_{m} \perp \{u_1, \dots, u_{m - 1}\}
\end{align*}
Le procédé va se terminer quand toutes les $2m - 1$ étapes seront faites. \\
Note: La notation $<u_1, \dots, u_j>$ est une façon plus courte d'écrire $\text{span}\{u_1, \dots, u_j\}$
\unnumsec{Valeurs et vecteurs propres d'une matrice carrée}

\subsection{Valeurs et vecteurs propres}
Soit $A \in M_{n \times n}(F)$
\begin{definition}
    $\lambda \in F$ valeur propre de $A \iff \exists \ u \in M_{n \times 1}(F), u \neq 0$ t.q. $Au = \lambda u$ \\
    Dans ce cas $u$ est dit le vecteur propre de $A$ associé à $\lambda$
\end{definition}
\begin{remark}
    Un vecteur propre n'est jamais nul, mais une valeur propre peut être nulle
\end{remark}
\begin{definition}
    Le polynôme caractéristique de $A$ est $\chi_A(\lambda) = \det(A - \lambda I)$
\end{definition}
\begin{theorem}
    Les valeurs propres de $A$ sont les racines du polynôme caractéristique de $A$, c-à-d
    \[
        \lambda_0 \ \text{valeur propre de} \ A \iff \chi_A(\lambda_0) = 0
    \]
\end{theorem}
\subsubsection{Comment calculer les vecteurs propres}
\begin{enumerate}
    \item Trouver les racines de $\chi_A(\lambda)$
    \item Trouver toutes les solutions linéairement indépendantes du système homogène $(A - \lambda_0 I)X = \mathbb{O}$
          pour toutes $\lambda_0$ valeurs propres de $A$.
    \item[] Note: Il se peut  qu'une valeur propre ait plusieurs vecteurs propres linéairement indépendants. % TODO: Put this note in the environment
\end{enumerate}
\subsubsection{Multiplicités et espace propre}
Grâce au théorème fondamentale de l'algèbre [\ref{theorem_fondamental_algebra}] on peut
exprimer le polynôme caractéristique à l'aide de ces racines, qui sont les valeurs propres
de $A$. Posons que $A$ a $m$  valeurs propres différentes entre elles. Alors
$\chi_A(\lambda) = (-1)^n(\lambda - \lambda_1)^{k_1}(\lambda - \lambda_2)^{k_2} \dots (\lambda - \lambda_m)^{k_m}
    \quad k_1 + k_2 + \dots + k_m = n, \ k_j \in \N$
\begin{definition}
    La multiplicité algébrique de la valeur propre $\lambda_j$ est le $k_j$ associé à $\lambda_j$
\end{definition}
\begin{definition}
    L'espace propre associé à $\lambda_j$ est
    $L_{\lambda_j} = \left\{ v \in M_{n \times 1} \mid Av = \lambda_j v \right\} \cup \left\{\mathbb{O} \in M_{n \times 1}\right\}$
\end{definition}
\begin{definition}
    La dimension de l'espace propre $L_{\lambda_j}$ est la multiplicité géométrique de $\lambda_j$
\end{definition}
\begin{lemma}
    $1 \leq \text{multiplicité géométrique de} \ \lambda_j \leq \text{multiplicité algébrique de} \ \lambda_j \leq n$
\end{lemma}

\subsection{Propriétés des valeurs propres} % Ajouté virgule avant alors et enlevé point en fin de phrse
Soit $A \in M_{n \times n}(\C)$
\paragraph{Propriété 1:} Si $B = \{v_1, \dots, v_n\}$ est un ensemble de vecteurs propres de $A$
associés à des valeurs propres distinctes, alors $B$ est linéairement indépendant.
\paragraph{Propriété 2:} Si $\lambda_0$ est une valeur propre de $A$ alors $(\lambda_0)^k$ est une
valeur propre de $A^k$, $k \in \N$
\paragraph{Propriété 3:} Si $\lambda_0$ est une valeur propre de $A$ et $A$ est inversible alors
$\frac{1}{\lambda_0}$ est une valeur propre de $A^{-1}$
\paragraph{Propriété 4:} Si $\det A = 0$ alors $\lambda_0 = 0$ est une valeur propre de $A$
\paragraph{Propriété 5:} $\det A = \lambda_1 \dots \lambda_n$ où $\lambda_j$ est une valeur propre de $A$
\paragraph{Propriété 6:} $\text{tr} A = \lambda_1 + \dots + \lambda_n$ où $\lambda_j$ est une valeur propre de $A$
\paragraph{Propriété 7:} $A$ et $A^T$ ont les mêmes valeurs propres.
\paragraph{Propriété 8:} Si $A$ est triangulaire alors les valeurs propres de $A$ sont sa diagonale.
\paragraph{Propriété 9:} Les valeurs propres d'une matrice hermitienne sont réelles
\paragraph{Propriété 10:} Si $A$ est hermitienne et que $v_1, v_2$ sont deux vecteurs propres de
$A$ associés à deux valeurs propres de $A$ distinctes, soit $\lambda_1$ et $\lambda_2$, alors $v_1 \perp v_2$
et $L_{\lambda_1} \perp L_{\lambda_2}$ par rapport au produit scalaire canonique dans $\C^n$
\paragraph{Propriété 11:} \label{property_11} Si $A$ est hermitienne, alors les vecteurs propres de $A$ forment une base orthonormale de $\C^n$

\subsection{Diagonalisation d'une matrice}
\begin{definition}
    Soit $A, P \in M_{n \times n}$ t.q. $P$ inversible, alors $A$ et $P^{-1}AP$ (ou $PAP^{-1}$) sont appellés semblables
\end{definition}
\begin{remark}
    $\det(P^{-1}AP) = \det(PAP^{-1}) = \det A$ \\
    De plus, les valeurs propres de $A$ et $P^{-1}AP$ coïncident.
\end{remark}
\begin{definition}
    $A \in M_{n \times n}$ est diagonalisable $\iff \exists \ P \in M_{n \times n}$ inversible t.q. $P^{-1}AP$ diagonale
\end{definition}
\begin{theorem}
    $A \in M_{n \times n}$ est diagonalisable $\iff A$ admet $n$ vecteurs propres linéairement indépendants \\
    Dans ce cas, on a que \[P = \begin{bmatrix}
            v_1 & v_2 & \dots & v_n
        \end{bmatrix} \ \text{où} \ v_1, v_2, \dots, v_n \ \text{les vecteurs propres de} \ A\] 
        De plus, on a que
        \[P^{-1}AP = \text{diag}\{\lambda_1, \dots, \lambda_n\} \ \text{où} \ \lambda_1, \dots, \lambda_n 
        \ \text{les valeurs propres de} \ A\] 
\end{theorem}
\begin{corollary}
    Si $A$ a $n$ valeurs propres distinctes, alors $A$ est diagonalisable
\end{corollary}
\begin{corollary}
    Une matrice hermitienne est toujours diagonalisable à cause de \ref{property_11}
\end{corollary}
\begin{corollary}
    \label{hermitienne_implies_unitaire_P}
    Si $A$ est hermitienne, alors $\exists \ P$ unitaire t.q. $P^{-1}AP = P^\dagger A P$ diagonale
\end{corollary}
\begin{remark}
    La preuve de \ref{hermitienne_implies_unitaire_P} utilise cette équivalence, soit
    \[
        \begin{matrix}
            P \ \text{unitaire} \iff & \text{les colonnes de} \ P \ \text{forment une base orthonormale dans} \\
                                     & \C^n \ \text{par rapport au produit scalaire canonique}
        \end{matrix}
    \]
\end{remark}

\subsection{Fonctions d'une matrice}
Soit $f(x) \in \C_n[x]$, ou $\C_n[x]$ est l'espace des polynômes complexes dont l'exposant est au plus $n$
\begin{definition}
    Soit $A \in M_{n \times n}(\C)$, alors $f(A) = \alpha_0 I + \alpha_1 A + \dots + \alpha_n A^n$
\end{definition}
Un autre cas considéré dans ce cours est $f(z) = e^z = \sum_{k = 0}^{\infty}\frac{z^k}{k!}$
\begin{definition}
    Soit $A \in M_{n \times n}(\C)$, alors $e^A = \sum_{k = 0}^{\infty}\frac{A^k}{k!} = I + A + \frac{A^2}{2!} + \frac{A^3}{3!} + \dots$
\end{definition}

\subsubsection{Comment calculer la fonction d'une matrice}
Il y a quatre éléments qui peuvent nous aider pour calculer $f(A)$
\begin{enumerate}
    \item $A$ est nilpotente d'indice $k$, c-à-d que $A^k = \mathbb{O}$, donc on doit seulement 
    calculer les $k + 1$ premiers termes de $f(A)$
    \item Il existe une formule pour les puissances de $A$ (souvent prouver par récurrence),
    alors il est possible d'utiliser cette formule pour simplifier le calcul
    \item $A = \text{diag}\{\lambda_1, \dots, \lambda_n \} \implies f(A) = \text{diag}\{f(\lambda_1), \dots, f(\lambda_n)\}$
    \item Si $A$ est diagonalisable alors $A = PDP^{-1}$ avec $D = \text{diag}\{\lambda_1, \dots, \lambda_n\}$ \\
    Alors $f(A) = Pf(D)P^{-1} = P\text{diag}\{f(\lambda_1), \dots, f(\lambda_n)\}P^{-1}$
\end{enumerate}
\begin{remark}
    Si $A$ est non diagonalisable et qu'il n'existe pas de formule pour les puissances de $A$,
    il faut calculer tous les termes de $f(A)$ pour obtenir la réponse. 
\end{remark}

\subsubsection{Propriétés de l'exponentielle d'une matrice}
\begin{enumerate}
    \item Comme $A$ commute avec $A^k \ \forall \ k \in \N$ alors $A$ commute avec $e^A$
    \item Si $AB = BA$, c-à-d $A$ et $B$ commute, alors $e^{A + B} = e^A e^B$
    \item $e^A$ inversible $\forall \ A \in M_{n \times n}$ avec $\left(e^A\right)^{-1} = e^{-A}$ 
    \item Si $A$ est hermitienne, alors $e^{iA}$ est unitaire
\end{enumerate}
\unnumsec{Espace dual d'un espace vectoriel}

\subsection{Fonctionnelles linéaires}
Soit $V$ un espace vectoriel avec les scalaires $F$
\begin{definition}
    Une forme linéaire, ou une fonctionnelle linéaire, est une application
    $f\colon V \to F$ qui, pour tout $u, v \in V$, $\scalaire{\alpha}$,
    satisfait les deux propriétés suivantes:
    \begin{align*}
        1. \quad f(u + v) = f(u) + f(v)& &2. \quad f(\alpha v) = \alpha f(v)
    \end{align*}
\end{definition}
\begin{remark}
    La fonctionnelle linéaire la plus simple est $\myfunc{f}{V}{F}{v}{0}$
\end{remark}
\begin{definition}
    Soit $f_1, f_2$ deux fonctionnelles linéaires. \\
    Définissons leur somme $f_1 + f_2$ comme telle:
    $(f_1 + f_2)(v) = f_1(v) + f_2(v) \ \forall \ v \in V $ \\
    Définissons la multiplication par un scalaire $\alpha f_1$ comme telle: 
    $(\alpha f_1)(v) = \alpha f_1(v) \ \forall \ v \in V$
\end{definition}
\begin{lemma}
    L'ensemble des fonctionnelles linéaires sur $V$ est lui-même un espace vectoriel
\end{lemma}
\begin{definition}
    L'espace des fonctionnelles sur $V$ est appelé l'espace dual de $V$ et est noté $V^*$
\end{definition}
\begin{theorem}
    Soit $B = \{b_1, \dots, b_n\}$ une base dans $V$ avec $[v]_B = \begin{pmatrix}
        v_1 & \dots & v_n \end{pmatrix}^T \ v \in V$. 
    Alors la base de $V^*$ est $B^* = \{\epsilon_1, \dots, \epsilon_n\}$ où 
    $\myfunc{\epsilon_j}{V}{F}{v}{v_j}$, et $v_j$ est la j-ème coordonnée de $v$ dans la base $B$.
\end{theorem}
\begin{remark}
    La définition de $\epsilon$ implique que $\epsilon_j(b_k) = \delta_{jk} \ \forall \ b_k \in B$
\end{remark}
\begin{corollary}
    $\dim V = n \implies \dim V^* = n$
\end{corollary}
\begin{corollary}
    $\left(\C^n\right)^* = \C^n$ et $\left(\R^n\right)^* = \R^n$ 
\end{corollary}

\subsection{Espace dual dans l'espace vectoriel muni d'un produit scalaire}
Soit $V$ un espace vectoriel muni d'un produit scalaire $\scpr{\cdot}{\cdot}$
\begin{definition}
    Le vecteur dual de $x \in V$ est un élément de l'espace dual $V^*$ notée $f_x$ ou $\hat{x}$. $f_x$ est la fonctionnelle linéaire qui projette les vecteurs sur $x$, c.-à-d.
    \[ \myfunc{f_x}{V}{F}{v}{\scpr{x}{v}} \]
\end{definition}
\begin{theorem} Chaque vecteur de $V$ a un unique vecteur dual, c.-à-d.
    \[\forall \ f \in V^* \ \exists! \ x \in V \ \text{t.q.} \ f = f_x\]
\end{theorem}
\subsubsection{Comment trouver le vecteur dual d'un élément de l'espace dual}
Soit $B = \{b_1, \dots, b_n\}$ une base \underline{orthonormale par rapport à $\scpr{\cdot}{\cdot}_V$} dans $V$ et $f \in V^*$ \\
Posons $f(b_j) = a_j \in \C^n$, alors la représentation du vecteur dual $x$ à $f$ dans la base $B$ est donnée par 
\[
[x]_B = \begin{pmatrix} (a_1)^* & \dots & (a_n)^* \end{pmatrix}^T 
\]
\begin{note}
    Dans le cas où le produit scalaire
    dans $V$ n'est pas le produit scalaire canonique, il est souvent plus simple d'utiliser une 
    méthode plus directe que de trouver une base orthonormale par rapport au produit scalaire dans $V$.
\end{note}
\begin{definition}
    Soit $B = \{u_1, \dots, u_n\}$ une base dans $V$, alors 
    une base $B^* = \{\epsilon_1, \dots, \epsilon_n\}$ dans $V^*$ est dite base dual à 
    $B \iff \epsilon_j(u_k) = \delta_{jk}$ 
\end{definition}
\subsection{Notation de Dirac}
Soit $V$ un espace vectoriel muni d'un produit scalaire.
\begin{definition}
    Dans la notation de Dirac, un élément $v \in V$ est noté $\ket{v} \in V$ et $\ket{\cdot}$ est appelé un ket. 
    Le vecteur dual de $\ket{v}$ est noté $\bra{v}$ et $\bra{\cdot}$ est appelé un bra. L'évaluation
    du vecteur dual $f_v = \ket{v}$ avec un vecteur $x \in V$ est notée $f_v(x) = f_v(\ket{x}) = \braket{v}{x}$
\end{definition}
\unnumsec{Opérateurs linéaires}
\begin{definition}
    Un opérateur $T$ est une application $T\colon V \to W$ avec $V$ et $W$ des espaces vectoriels
\end{definition}
\begin{definition}
    Un opérateur $T$ est linéaire $\iff T(\alpha u + \beta v) = \alpha T(u) + \beta T(v) \ \forall \ u, v \in V et \ \alpha, \scalaire{\beta}$ 
\end{definition}
\begin{definition}
    Soient $T_1 \colon V \to W$ et $T_2 \colon W \to U$. La composition de $T_1$ et $T_2$ est noté
    $T_2 T_1$ et l'évaluation est $T_2 T_1 (v) = T_2(T_1(v))$ avec $T_1 T_2 \colon V \to U$
\end{definition}
\begin{remark}
    La composition de deux opérateurs n'est généralement pas commutatif, mais est associatif
\end{remark}
\begin{definition}
    Soit $T_1, T_2 \colon V \to V$. Le commutateur de $T_1$ et $T_2$ est 
    $[T_1, T_2] = T_1T_2 - T_2T_1$ avec $[T_1, T_2](v) = T_1T_2v - T_2T_1v$
\end{definition}

\subsection{La matrice d'un opérateur linéaire}
\begin{theorem}
    Soient $V, W$ deux espaces vectoriels où $\dim_\C V = n$ et $\dim_\C W = m$ \\
    Soient $B_V, B_W$ deux bases respectives de $V$ et $W$, alors 
    \[
    T\colon V \to W \ \text{est linéaire} \iff \exists \ M \in M_{m \times n}(\C) \ \text{t.q.} \ 
    \forall \ v \in V \ M[v]_{B_V} = [Tv]_{B_W}
    \]  
    De plus, si $B_V = \{v_1, \dots, v_n\}$ alors la matrice $M$ est donnée par 
    \[
    M = \left[ \left[ Tv_1 \right]_{B_W}, \dots, \left[ Tv_n \right]_{B_W} \right]
    \]
\end{theorem}
\begin{definition}
    La matrice $M$ t.q. $M[v]_{B_V} = [Tv]_{B_W}$ est dite la matrice de l'opérateur $T$
    dans les deux bases $B_V, B_W$, notée $M = [T]_{B_V, B_W}$. Si $V = W$ et $B_V = B_W$
    alors on note $M = [T]_{B_V}$
\end{definition}

\subsection{Représentation d'un opérateur linéaire dans une nouvelle base}
\begin{theorem}
    Soit $T\colon V \to W$ un opérateur linéaire. \\
    Soient $B_V, B_V^\prime$ deux bases dans $V$ et $B_W, B_W^\prime$ deux bases dans $W$, alors
    $[T]_{B_V^\prime, B_W^\prime} = P_{B_W^\prime}^{B_W} [T]_{B_V, B_W} P^{B_V^\prime}_{B_V}$ \\
    Si $V = W$ et $B_V = B_W$, alors $[T]_{B_V} = P_{B_V^\prime}^{B_V} [T]_{B_V} P^{B_V^\prime}_{B_V}$
\end{theorem}
\begin{theorem}
    Soient $T_1, T_2 \colon V \to V$ des opérateurs linéaires, $B_V$ une base dans $V$ et $\scalaire{\alpha}$, alors 
    les opérateurs suivants sont linéaires, soit $T_1 + T_2, \ \alpha T_1, \ T_1T_2, \ [T_1, T_2]$.
    Les matrices des opérateurs suivants sont 
    \begin{align*}
        1) \quad &[T_1 + T_2]_{B_V} = [T_1]_{B_V} + [T_2]_{B_V}& &2) \quad [\alpha T_1]_{B_V} = \alpha [T_1]_{B_V} \\
        3) \quad &[T_1T_2]_{B_V} = [T_1]_{B_V}[T_2]_{B_V}& &4) \quad \left[[T_1, T_2]\right]_{B_V} = \left[[T_1]_{B_V}, [T_2]_{B_V}\right]
    \end{align*}
\end{theorem}

\subsection{Valeurs et vecteurs propres d'un opérateur}
\begin{definition}
    Soit $T\colon V \to V$ un opérateur linéaire, alors 
    \[v \in V \ \text{vecteur propre de} \ T \iff \exists \ \scalaire{\lambda} \ \text{t.q.} \ Tv = \lambda v\]
    $\lambda$ est appelé la valeur propre de $T$
\end{definition}
\begin{theorem}
    Si $\dim V \in \N$, alors $\forall \ v \in V$ vecteur propre de $T$, $[v]_B$ vecteur propre de $[T]_B$ \\
    Cela va de même pour les valeurs propres de $T$
\end{theorem}
\begin{definition}
    L'ensemble des valeurs propres d'un opérateur s'appelle le spectre de l'opérateur
\end{definition}
\begin{theorem}
   $\text{Ker}(T) \neq \{0_V\} \iff \lambda = 0$ valeur propre de $T$, ou $\text{Ker}(T) = \left\{ v \in V \mid Tv = 0_W \right\}$
\end{theorem}
\unnumsec{Opérateurs inverses, hermitiens et unitaires}

\subsection{Opérateurs inverses}
Soit $T\colon V \to W$ un opérateur linéaire
\begin{definition}
    $T$ inversible $\iff \forall \ w \in W \ \exists! \ v \in V$ t.q. $Tv = w$
\end{definition}
\begin{definition}
    L'opérateur inverse de $T$ est $\myfunc{T^{-1}}{W}{V}{w}{v}$ 
\end{definition}
Soient $B_V, B_W$ des bases dans $V, W$ respectivement 
\begin{theorem}
    $T$ inversible $\iff \dim V = \dim W$ et $[T]_{B_V, B_W}$ inversible \\
    Dans ce cas, $T^{-1} = [T^{-1}]_{B_W, B_V} = \left([T]_{B_V, B_W}\right)^{-1}$
\end{theorem}
\begin{corollary}
    $T$ inversible $\iff [T]_{B_V, B_W}$ carré
\end{corollary}
\begin{remark}
    $[T]_{B_V, B_W}$ carré $\implies \dim V = \dim W$
\end{remark}

\subsubsection{Propriétés de l'opérateur inverse}
Soient $T, S\colon V \to W$ des opérateurs linéaires inversibles
\begin{enumerate}
    \item $T^{-1}\colon W \to V$ est linéaire
    \item $[T^{-1}]_{B_W, B_V} = \left([T]_{B_V, B_W}\right)^{-1}$
    \item $\left(T^{-1}\right)^{-1} = T$
    \item $(TS)^{-1} = S^{-1}T^{-1}$
    \item $T$ inversible $\iff \text{Ker}(T) = \{0_W\}$
\end{enumerate}

\subsection{Opérateurs hermitiens}
Soit $V$ un espace vectoriel muni d'un produit scalaire $\scpr{\cdot}{\cdot}$ \\
Soit $B$ une base orthonormale dans $V$ par rapport au produit scalaire dans $V$ \\
Soit $A\colon V \to V$ un opérateur linéaire
\begin{definition}
    $A$ hermitien $\iff [A]_B$ hermitien 
\end{definition}
\begin{theorem}
    Peu importe la base $B$ choisie, si $A$ hermitien alors $[A]_B$ hermitien
\end{theorem}
\begin{remark}
    Le théorème nous permet de justifier la définition d'un opérateur hermitien, puisque 
    le théorème prouve que la définition ne dépend pas de la base choisie
\end{remark}

\subsubsection{Propriétés de l'opérateur hermitien}
Soient $A, T\colon V \to V$. Les opérateurs suivants sont hermitiens
\begin{align*}
    1. \quad A + T& &2. \quad \alpha A \ \text{si} \ \alpha \in \R& &3. \quad AT \ \text{si} \ A,T \ \text{commute}& &4. \quad A^n, T^n& &5. \quad A^{-1} \ \text{si} \ A \ \text{inversible} 
\end{align*}
De plus, un opérateur hermitien est indépendant dans le produit scalaire. 
\begin{lemma}
    Soient $v_1, v_2 \in V$, alors $\scpr{v_1}{Av_2}_V = \scpr{Av_1}{v_2}_V$
\end{lemma}
\noindent
Dans la notation de Dirac, on note cette propriété comme telle: $\bra{v_1}A\ket{v_2}$,
ce qui indique que $A$ peut être appliqué à $\bra{v_1}$ ou $\ket{v_2}$

\subsection{Opérateurs unitaires}
Soit $V$ un espace vectoriel muni d'un produit scalaire $\scpr{\cdot}{\cdot}$ \\
Soit $B$ une base orthonormale dans $V$ par rapport au produit scalaire dans $V$ \\
Soit $A\colon V \to V$ un opérateur linéaire
\begin{definition}
    $A$ unitaire $\iff [A]_B$ unitaire 
\end{definition}
\begin{remark}
    Comme dans le cas de l'opérateur hermitien, cette définition ne dépend pas de la base $B$ choisie
\end{remark}

\subsubsection{Propriétés de l'opérateur unitaire}
\begin{lemma}
    Si $A\colon V \to V$ unitaire et $v_1, v_2 \in V$, alors $\scpr{v_1}{v_2}_V = \scpr{Av_1}{Av_2}_V $
\end{lemma}

\end{document}
