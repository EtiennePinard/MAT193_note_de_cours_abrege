\documentclass[11pt]{article}

\usepackage[T1]{fontenc}
\usepackage[french]{babel}
\usepackage[margin = 2cm]{geometry}
\usepackage[shortlabels]{enumitem}
\usepackage[hidelinks]{hyperref}
\usepackage{math_and_phy_utils}
\usepackage[pdftex]{graphicx}
\usepackage{tikz}
\usetikzlibrary{angles,quotes, matrix,arrows,decorations.pathmorphing}

\usepackage{fancyhdr}
\setlength{\headheight}{15.2pt}
% The footnote code is from https://tex.stackexchange.com/a/74992
\newcommand{\fancyfootnotetext}[2]{%
  \fancypagestyle{dingens}{%
    \fancyfoot[C]{}
    \fancyfoot[L]{\footnotemark[#1]\footnotesize #2}%
    \fancyfoot[R]{\thepage}
  }%
  \thispagestyle{dingens}%
}

\pagestyle{fancy}
\fancyhead{}
\fancyfoot[c]{\thepage}

\usepackage{amsmath}
\usepackage{amssymb}
\usepackage{amsthm}

\theoremstyle{plain}
\newtheorem{theorem}{Théorème}[section]
\newtheorem{lemma}[theorem]{Lemme}
\newtheorem{corollary}[theorem]{Corollaire}

\theoremstyle{definition}
\newtheorem{definition}[theorem]{Définition}

\theoremstyle{remark}
\newtheorem*{remark}{Remarque}
\newtheorem*{note}{Note}
\AfterEndEnvironment{note}{\noindent\ignorespaces}
\AfterEndEnvironment{remark}{\noindent\ignorespaces}

\newcommand{\unnumsec}[1]{
    \refstepcounter{section}
    \addcontentsline{toc}{section}{Chapitre \thesection: #1}
    \fancyhead[c]{Chapitre \thesection: #1}
    \section*{Chapitre \thesection: #1}
}

\begin{document}

% This is to disable the spacing before a colon of the french babel package. Use \shorthandon{:} to renable it
\shorthandoff{:}

% Gotten from https://tex.stackexchange.com/a/325032
\begin{titlepage}
  \centering
  \vspace*{3in}

  \vspace*{0.5cm}

  \Huge
  \textmd{\textbf{Algèbre linéaire}}\\
  \vspace{0.1in} \LARGE \textmd{\textbf{Notion de base}} \\
  \vspace{0.1in}\large{Basé sur le cours MAT193}

  \vspace*{\fill}
  \large Étienne Pinard \\
  Compilé le \today

\end{titlepage}

\pagebreak

\fancyhead[c]{Table des matières}
\tableofcontents

\unnumsec{Chapitre 1: Nombres Complexes}

\subsection{Définition des nombres complexes}
Rappel: les unités de $\R$ sont $1$ et $-1$ 
\begin{definition}
    $i$ est l'unité imaginaire telle que $i^2 = -1$ 
\end{definition}
Alors, il est possible de former un nouveau ensemble qui a a les unités $1, -1, i$ et $\scalaire{\alpha} \in \R$
\begin{definition}
    L'ensemble qui a a les unités $1, -1, i$ et $\scalaire{\alpha} \in \R$ est appellé nombre complexes et est noté $\C$
\end{definition}

\subsection{Représentation des nombres complexes}
Soit $x, y, r, \theta \in \R$ et $z \in \C$. Il a trois façons principales de représenter $z$
\begin{enumerate}
    \item \underline{Forme algégrique (cartésienne)}: $z = x + iy$
    \item \underline{Forme polaire}: $z = r(\cos(\theta)+i\sin(\theta))$
    \item \underline{Forme exponentielle}: $z = re^{i\theta}$
\end{enumerate}

\subsection{Changement de forme: Algèbrique à polaire/exponentielle}
Soit $z = x + iy = r(\cos(\theta)+i\sin(\theta)) = re^{i\theta}$ 
\begin{definition}
    On appelle $r$ le module de $z$ noté $|z|$ avec $r = |z| = \sqrt{x^2 + y^2}$
\end{definition}
\begin{definition}
On appelle $\theta$ l'argument de $z$ noté $\text{Arg}(z)$ avec \[\theta = Arg(z) =
    \begin{cases}
        \arctan(\frac{y}{x})       & x > 0                            \\
        \arctan(\frac{y}{x}) - \pi & x < 0                            \\
        \text{sign}(y)\frac{\pi}{2}           & x = 0, \ \text{sign}(y) \ \text{est la signe de y}
    \end{cases}\]
\end{definition}
\begin{remark}
    L'argument non unique de $z$ est $\arg(z) = \text{Arg}(z) + 2\pi k, \ k \in \Z$
\end{remark}

\subsection{Opérations propres aux nombres complexes}
Soit $z \in \C, \ x, y \in \R, \ \text{t.q.} \ z = x + iy$ 
\begin{definition}
    Les parties réel et imaginaire de $z$ sont $\Real{z} = x \in \R$ et $\Ima{z} = y \in \R$
\end{definition}
\begin{definition}
    Le conjugée de $z$ est $z^* = \bar{z} = x - iy = \Real{z} - i\Ima{z}$
\end{definition}

\subsubsection{Propriétés des opérations}
Le conjugée et le module sont distributif sur l'addition, la multiplication et la division
\[
    \begin{matrix}
        \left(z_1 + z_2\right)^* = z_1^* + z_2^* & \left(z_1z_2\right)^* = z_1^{*} z_2^* & \left( \frac{z_1}{z_2}\right)^{*} = \frac{z_1^*}{z_2^*} \\[0.5em]
        \left|z_1 + z_2\right| = |z_1| + |z_2|   & \left|z_1z_2\right| = |z_1||z_2|      & \left| \frac{z_1}{z_2}\right| = \frac{|z_1|}{|z_2|}
    \end{matrix}
\]
Le conjugée de $z^*$ est $z$, soit $\left(z^*\right)^* = z$ \\
Le conjugée $z$ et le module de $z$ est relié par $zz^* = |z|^2$ \\
L'argument de $z$ à des propriétés semblabes aux propriétés logarithmiques
\[
    \begin{matrix}
        \arg(z_1z_2) = \arg(z_1) + \arg(z_2) & \arg\left( \frac{z_1}{z_2}\right) = \arg(z_1) - \arg(z_2)
    \end{matrix}
\]

\subsection{Racines entières}
Soit $z = re^{i\theta},\ n \in \N$.
\[
    \sqrt[n]{z} = \sqrt[n]{r} \left( e^{i \left( \frac{\theta + 2\pi k}{n} \right) } \right), \ k = 0, \dots, n - 1
\]
On peut aussi calculer les racines de $z$ de manières récursives. \\
Si $w$ est une racine de $z$, alors on a
\[
    w_k = \begin{cases}
        w_{k - 1}e^{ \frac{2\pi i}{n} }                    & k > 0 \\[0.5em]
        \sqrt[n]{r} \left( e^{i \frac{\theta}{n} } \right) & k = 0
    \end{cases}
\]
\begin{remark}
    Deux racines entières consécutives sont séparées par un angle de $\frac{2\pi}{n}$
\end{remark}

\subsection{Exponentielle  et logarithme}
Soit $z \in \C, \ x, y \in \R, \ \text{t.q.} \ z = x + iy$ \\
L'exponentielle de $z$ est
\[
    e^z = e^{x + iy} = e^x e^{iy} = e^x (cos(y) + isin(y))
\]
Le logarithme de $z$ est
\begin{align*}
    \ln(z) = \log(z) & = \ln\left(|z|e^{i(\theta + 2\pi k)}\right)      \\
                     & = \ln|z| + i(\theta + 2\pi k), \ k \in \Z        \\
                     & = \ln|z| + i(\text{Arg}(z) + 2\pi k), \ k \in \Z \\
                     & = \ln|z| + i\arg(z)                              \\
\end{align*}

\subsection{Puissance Complexe}
Soit $z \in \C$ et $w \in \C, \ x, y \in \R \ \text{t.q.} \ w = x + iy$ 
\[
    z^w = \left( e^{\ln(z)}\right)^{w} = e^{\ln(z)w}
    = e^{\left(\ln|z| + i\arg(z)\right)\left(x + iy\right)}
    = |z|^{x} e^{-y\arg(z)} e^{ i \left(x\arg(z) + y\ln|z|\right) }
\]
Si $y \neq 0$ alors $z^w$ prend une infinité de valeurs distinctes, puisque $e^{-y\arg(z)} = e^{-y\text{Arg}(z) + 2\pi yk}, \; k \in \Z$ \\
Si $y = 0 \implies w \in \R$ alors le nombre de valeurs de $z^w$ dépend dans quelle ensemble $x$ est. \\
Si $x \in \Q$, alors $x = \frac{m}{n}, \ m \in \Z, \ n \in \N \implies \exists \ k \in \Z \ \text{t.q.} \ kx \in \Z \implies z^w$ a $n$ valeurs distinctes \\
Si $x \in \R \backslash \Q \implies \not \exists \ k \in \Z \ \text{t.q.} \ kx \in \Z \implies z^w$ a une infinité de valeurs distinctes \\
Dans ce dernier cas, si $z, w \in \R$ alors, par convention, on prend $k = 0$ pour que $z^w \in \R$.

\subsection{Théorème fondamental de l'algèbre}
\label{theorem_fondamental_algebra}
Soit $p(x) = a_n x^n + a_{n-1} x^{n-1} + \dots + a_1 x + a_0, \; a_j \in \C, \; a_n \neq 0, \; n \in N$ \\
Alors on peut écrire $p(x)$ en terme de ses racines $\{x_1, x_2, \dots ,x_m\}, \; x_j \in \C $, soit
\[ p(x) = a_n(x - x_1)^{k_1} (x - x_2)^{k_2} \dots (x - x_m)^{k_m} \]
où $k_j$ est la multiplicité de la racine $x_j$ avec $k_1 + k_2 + \dots + k_m = n$ \\
Alors $p(x)$ a exactement $n$ racines complexes en comptant les multiplicités.
\section*{Chapitre 2: Matrices}

\subsection*{Définition}
\begin{itemize}
    \item[] Une matrice de type $m \times n$ sur un corp (dans notre cas sur $\mathbb{R}$ ou sur $\mathbb{C}$) est noté
          $A_{m \times n}$ et contient $m$ lignes et $n$ colonnes. \begin{equation*}
              A_{m \times n} = \begin{pmatrix}
                  a_{11} & a_{12}      & \dots  & a_{1n} \\
                  a_{21} & a_{22}      & \dots  & a_{2n} \\
                  \vdots & \phantom{a} & \ddots & \vdots \\
                  a_{m1} & a_{m2}      & \dots  & a_{mn}
              \end{pmatrix}
          \end{equation*}
    \item[] La matrice composé uniquement de 0 pour un certain type $m \times n$ est écrite $\mathbb{O}$
    \item[] $A_{m \times n}$ est une matrice carré si $m = n$.
    \item[] La matrice identité de type $n$ est écrite $I_n = I = \begin{pmatrix}
                  1      & 0           & \dots  & 0      \\
                  0      & 1           & \dots  & 0      \\
                  \vdots & \phantom{a} & \ddots & \vdots \\
                  0      & 0           & \dots  & 1
              \end{pmatrix}$
    \item[] Soit $A \in M_n(\mathbb{C})$. Alors on dit que $A$ est \begin{enumerate}[itemsep = 0.5em]
              \item triangulaire supérieur si $A = \begin{pmatrix}
                            a_{11} & a_{12}      & \dots  & a_{1n} \\
                            0      & a_{22}      & \dots  & a_{2n} \\
                            \vdots & \phantom{a} & \ddots & \vdots \\
                            0      & 0           & \dots  & a_{nn}
                        \end{pmatrix}$
              \item triangulaire inférieur si $A = \begin{pmatrix}
                            a_{11} & 0           & \dots  & 0      \\
                            a_{21} & a_{22}      & \dots  & 0      \\
                            \vdots & \phantom{a} & \ddots & \vdots \\
                            a_{n1} & a_{n2}      & \dots  & a_{nn}
                        \end{pmatrix}$
              \item triangulaire si $A$ est triangulaire supérieur ou triangulaire inférieur
              \item Diagonale si $A = \begin{pmatrix}
                            a_{11} & 0           & \dots  & 0      \\
                            0      & a_{22}      & \dots  & 0      \\
                            \vdots & \phantom{a} & \ddots & \vdots \\
                            0      & 0           & \dots  & a_{nn}
                        \end{pmatrix} = \text{diag}\{ a_{11}, a_{22}, \dots, a_{nn}\}$ \\[0.5em]
                    Noté que $I_n = \text{diag}\{ 1, 1, \dots, 1 \}$ ainsi que $\mathbb{O}_n = \text{diag}\{ 0, 0, \dots, 0 \}$
          \end{enumerate}
\end{itemize}

\subsection*{Opérations matricielles}
\begin{itemize}
    \item[] \textbf{Addition de matrices} \begin{itemize}
              \item[] L'addition de deux matrices $A$ et $B$ existe seulement si $A$ et $B$ sont du même type.
                    L'addition est commutatif, associatif, possède un élément neutre et possède un inverse.
          \end{itemize}
    \item[] \textbf{Multiplication par un scalaire} \begin{itemize}
              \item[] La multication par un scalaire existe toujours pour une matrice $A$. La multication
                    par un scalaire est associatif, distributif et commutatif.
          \end{itemize}
    \item[] \textbf{Produit matricielle} \begin{itemize}
              \item[] Soit $A = (a_{ij}) \in M_{m \times n}(\mathbb{C})$ et $B = (b_{ij}) \in M_{r \times s}(\mathbb{C})$. Alors
                    le produit matricielle $AB$ existe si $n = r$. Dans ce cas, on a \begin{equation*}
                        (AB)_{ij} = \sum_{s = 1}^{n} a_{is} b_{sj}
                    \end{equation*}
                    En gros, l'élément à la position $i, j$ de $AB$ est le produit scalaire entre la i-ème ligne
                    de A et la j-ème colonne de B.
          \end{itemize}
    \item[] \textbf{Propriétés du produit matricielle} \begin{itemize}
              \item[] Soit $\alpha, \beta$ des scalaires et $A, B, C$ des matrices dont le produit matricielle entre eux existe. Alors \begin{enumerate}
                        \item $(AB)C = A(BC)$, produit matricielle est associatif
                        \item $AB \neq BA$, généralement, le produit matricielle n'est pas commutatif
                        \item $A(B + C) = AB + AC$, distributivité par la gauche
                        \item $(A + B)C = AC + BC$, distributivité par la droite
                        \item $AI = IA = A$, l'identité est l'élément neutre du produit matricielle
                    \end{enumerate}
          \end{itemize}
    \item[] \textbf{Transposition} \begin{itemize}
              \item[] La transposé de $A = (a_{ij})_{m \times n}$ est $A^{T} = (a_{ji})_{n \times m}$.
                    Pour une matrice carré, c'est une rotation des anti-diagonale par rapport à la grande diagonale.
          \end{itemize}
    \item[] \textbf{Conjugée hermitien} \begin{itemize}
              \item[] Le conjugué hermitien de $A = (a_{ij})_{m \times n}$ est $A^{\dagger} = \left(A^T\right)^\star = \left(A^\star\right)^T = (a_{ji}^\star)_{n \times m}$.
                    Le conjugué hermitien est aussi appellé la transposé conjugué.
          \end{itemize}
    \item[] \textbf{Propriétés transposé et conjugué hermitien} \begin{align*}
               & \left( A^T \right)^T = A               &  & \left( A^\dagger \right)^\dagger = A                     \\
               & \left( \alpha A \right)^T = \alpha A^T &  & \left( \alpha A \right)^\dagger = \alpha^\star A^\dagger \\
               & \left( A + B \right)^T = A^T + B^T     &  & \left( A + B \right)^\dagger = A^\dagger + B^\dagger     \\
               & \left( AB \right)^T = B^TA^T           &  & \left( AB \right)^\dagger = B^\dagger A^\dagger
          \end{align*}
    \item[] \textbf{Définition} \begin{enumerate}
              \item Si $A^T = A$, alors $A$ est dite symétrique. Si $A^T = -A$, alors $A$ est dite anti-symétrique.
              \item Si $A$ est une matrice carré tel que $AA^T = A^TA = I$, alors $A$ est appellé orthogonal.
              \item Si $A^\dagger = A$, alors $A$ est dite hermitienne. Si $A^\dagger = -A$, alors $A$ est dite anti-hermitienne.
              \item Si $A$ est une matrice carré tel que $AA^\dagger = A^\dagger A = I$, alors $A$ est appellé unitaire.
          \end{enumerate}
    \item[] \textbf{Commutateur de matrice} \begin{itemize}
              \item[] Soit $A, B \in M_n(\mathbb{C})$, alors le commutateur de $A$ et $B$ est la matrice \begin{equation*}
                        \left[ A, B \right] = AB - BA
                    \end{equation*}
          \end{itemize}
    \item[] \textbf{Propriétés du commutateur} \begin{enumerate}
              \item $[A, B] = -[B, A]$
              \item $[\alpha A, B ] = \alpha [A, B] $
              \item $[A + B, C] = [A, C] + [B, C]$
              \item $[A, B]^T = -[A^T, B^T]$
              \item $[[A, B], C] + [[C, A], B] + [[B, C], A] = \mathbb{O}$, l'identité de Jacobi
          \end{enumerate}
    \item[] \textbf{Trace d'une matrice} \begin{itemize}
              \item[] Soit $A = (a_{ij})_{n \times n}$. Alors la trace de $A$ est \begin{equation*}
                        \text{tr} A =  \sum_{i = 0}^{n} a_{ii}
                    \end{equation*}
          \end{itemize}
    \item[] \textbf{Propriétés de la trace} \begin{enumerate}
              \item $\text{tr}\{ AB \} = \text{tr}\{ BA \}$
              \item $\text{tr}\{ A + B \} = \text{tr}A + \text{tr}B$
              \item $\text{tr}\{ \alpha A \} = \alpha \text{tr}A $
              \item $\text{tr}\{ A^T \} = \text{tr}A$
              \item $\text{tr}\{ [A, B] \} = \text{tr}\{ AB - BA \} = \text{tr}\{ AB \} - \text{tr}\{BA\} = \text{tr}\{ AB \} - \text{tr}\{AB\} = 0$
          \end{enumerate}
\end{itemize}

\subsection*{Déterminant d'une matrice carré}
\begin{itemize}
    \item[] \textbf{Défintion du déterminant par récurrence} \begin{itemize}
              \item[] Le déterminant de $A \in M_n(\mathbb{C})$ noté $\det A = |A|$ est une nombre complexe
                    qui peut-être définie par récurrence sur n comme suit \begin{equation*}
                        \det A = \begin{cases}
                            a_{11},                                   & \text{si } n = 1 \\
                            \sum_{i = 1}^{n} (-1)^{j + i}a_{ji}m_{ji} & \text{si } n > 1
                        \end{cases}
                    \end{equation*}
              \item[] Ceci est l'expansion par la jème ligne, $1 \leq j \leq n$. Le terme $m_{ji}$
                    est le déterminant de la sous-matrice qu'on obtient si on biffe la ligne $j$ et la
                    colonne $i$. Si $M_{ji}$ est la sous-matrice obtenue en biffant la ligne $j$ et
                    la colonne $i$ alors $m_{ji} = \det M_{ji}$.
              \item[] Dans la première équation, le déterminant des sous-matrices était multiplié par la jème ligne. Il est
                    équivalent de développer le déterminant suivant la jème colonne, ce qui donne cette équation \begin{equation*}
                        \det A = \begin{cases}
                            a_{11},                                   & \text{si } n = 1 \\
                            \sum_{i = 1}^{n} (-1)^{i + j}a_{ij}m_{ij} & \text{si } n > 1
                        \end{cases}
                    \end{equation*}
          \end{itemize}
    \item[] \textbf{Défintion du déterminant par les permutations} \begin{itemize}
              \item[] Le déterminant de $A$ peut aussi être définie de cette façons \begin{equation*}
                        \det A = \sum_{\sigma \in S_n} \text{sign}(\sigma) \prod_{i = 1}^{n}a_{j \sigma_j}
                    \end{equation*}
                    Ici $\sigma = \begin{pmatrix}
                            \sigma_1 & \sigma_2 & \dots & \sigma_n
                        \end{pmatrix}$ et représente une permutation de l'ensemble $\{1, \ 2, \dots, \ n  \}$.
                    $\sigma_j$ est le jème élément de cette permutation. $S_n$ est l'ensemble de tous ces
                    permutations. $\text{sign}(\sigma)$ est la signature de $\sigma$ et est égale à
                    $(-1)^{\text{nb de désorde de } \sigma}$. Le désorde de $\sigma$ est le nombre de couple
                    $(\sigma_j, \sigma_k)$ ou $j < k$ mais $\sigma_j > \sigma_k$.
          \end{itemize}
    \item[] \textbf{Propriétés du déterminant} \begin{itemize}
              \item[] Soit $A, B \in M_n(\mathbb{C}), \ \alpha \in \mathbb{C}$ \begin{enumerate}
                        \item $\det A^T = \det A$
                        \item $\det(\alpha A) = \alpha^n \det(A)$
                        \item $\det (A^\star) = (\det A)^\star$
                        \item $\det A^\dagger = (\det A)^\star$
                        \item $\det(AB) = \det(A) \det(B)$
                        \item $\det(A^m) = (\det A)^m$
                        \item $\det(A + B) \neq \det A + \det B$
                        \item $\det(\text{diag}\{a_1, \ a_2, \ldots, a_n \}) = a_1 a_2 \dots a_n$
                        \item Si $A$ est triangulaire alors $\det A = a_{11} a_{22} \dots a_{nn}$
                    \end{enumerate}
              \item[] Soit $A = \begin{pmatrix}
                            L_1    \\
                            L_2    \\
                            \vdots \\
                            L_n
                        \end{pmatrix} = \arraycolsep=0.3\arraycolsep \begin{pmatrix}
                            C_1 & C_2 & \ldots & C_n
                        \end{pmatrix}$ ou $L_j \in M_{1 \times n}(\mathbb{C}), \ C_j \in M_{n\times 1}(\mathbb{C})$ \begin{enumerate}[itemsep = 0.5em]
                        \item $\det \begin{pmatrix}
                                      L_1   \\
                                      \dots \\
                                      L_j   \\
                                      \dots \\
                                      L_k   \\
                                      \dots \\
                                      L_n
                                  \end{pmatrix} = -\det \begin{pmatrix}
                                      L_1   \\
                                      \dots \\
                                      L_k   \\
                                      \dots \\
                                      L_j   \\
                                      \dots \\
                                      L_n
                                  \end{pmatrix} \\[0.5em] \det \arraycolsep=0.3\arraycolsep \begin{pmatrix}
                                      C_1 & \ldots & C_j & \ldots & C_k & \ldots & C_n
                                  \end{pmatrix} = -\det \begin{pmatrix}
                                      C_1 & \ldots & C_k & \ldots & C_j & \ldots & C_n
                                  \end{pmatrix}$
                        \item $\det \begin{pmatrix}
                                      L_1   \\
                                      \dots \\
                                      L_j   \\
                                      \dots \\
                                      L_j   \\
                                      \dots \\
                                      L_n
                                  \end{pmatrix} = 0 \\[0.5em] \det \arraycolsep=0.3\arraycolsep \begin{pmatrix}
                                      C_1 & \ldots & C_j & \ldots & C_j & \ldots & C_n
                                  \end{pmatrix} = 0$
                        \item $\det \begin{pmatrix}
                                      L_1        \\
                                      \dots      \\
                                      \alpha L_j \\
                                      \dots      \\
                                      L_n
                                  \end{pmatrix} = \alpha \det \begin{pmatrix}
                                      L_1   \\
                                      \dots \\
                                      L_j   \\
                                      \dots \\
                                      L_n
                                  \end{pmatrix} \\[0.5em] \det \arraycolsep=0.3\arraycolsep \begin{pmatrix}
                                      C_1 & \ldots & \alpha C_j & \ldots & C_n
                                  \end{pmatrix} = \alpha \det \begin{pmatrix}
                                      C_1 & \ldots & C_j & \ldots & C_n
                                  \end{pmatrix}$
                        \item $\det \begin{pmatrix}
                                      L_1   \\
                                      \dots \\
                                      L_j   \\
                                      \dots \\
                                      L_k   \\
                                      \dots \\
                                      L_n
                                  \end{pmatrix} = \det \begin{pmatrix}
                                      L_1              \\
                                      \dots            \\
                                      L_j + \alpha L_k \\
                                      \dots            \\
                                      L_k              \\
                                      \dots            \\
                                      L_n
                                  \end{pmatrix} \\[0.5em] \det \arraycolsep=0.3\arraycolsep \begin{pmatrix}
                                      C_1 & \ldots & C_j & \ldots & C_k & \ldots & C_n
                                  \end{pmatrix} = \det \begin{pmatrix}
                                      C_1 & \ldots & C_j + \alpha C_k & \ldots & C_k & \ldots & C_n
                                  \end{pmatrix}$
                        \item $\det \begin{pmatrix}
                                      L_1   \\
                                      \dots \\
                                      M + N \\
                                      \dots \\
                                      L_n
                                  \end{pmatrix} = \det \begin{pmatrix}
                                      L_1   \\
                                      \dots \\
                                      M     \\
                                      \dots \\
                                      L_n
                                  \end{pmatrix} + \det \begin{pmatrix}
                                      L_1   \\
                                      \dots \\
                                      N     \\
                                      \dots \\
                                      L_n
                                  \end{pmatrix}  \\[0.5em] \det \arraycolsep=0.3\arraycolsep \begin{pmatrix}
                                      C_1 & \ldots M + N & \ldots & C_n
                                  \end{pmatrix} = \det \begin{pmatrix}
                                      C_1 & \ldots M & \ldots & C_n
                                  \end{pmatrix} + \det \begin{pmatrix}
                                      C_1 & \ldots N & \ldots & C_n
                                  \end{pmatrix}$
                    \end{enumerate}
          \end{itemize}
\end{itemize}

% Temp break for current formatting. Will change if formatting changes
\break
\subsection*{Matrice inverses}
\begin{itemize}
    \item[] $A \in M_{n \times n} \text{ est inversible} \iff \exists B \text{ t.q } AB = BA = I$. Alors $B$ est noté $A^{-1}$
    \item[] \textbf{Propriétés de l'inverse} \begin{itemize}
              \item[] Soit $A \in M_{n \times n}(\mathbb{C})$ inversible \begin{enumerate}
                        \item $AB = \mathbb{O} \implies B = \mathbb{O}$
                        \item Si $AC = BA = I \implies B = BI = B(AC) = (BA)C = IC = C $, l'inverse à gauche et à droite sont égaux
                        \item $(A^{-1})^{-1} = A$, \quad $(A^T)^{-1} = (A^{-1})^{T}$, \quad $(A^\dagger)^{-1} = (A^{-1})^{\dagger}$
                        \item $(AB)^{-1} = B^{-1}A^{-1}$
                    \end{enumerate}
          \end{itemize}
    \item[] \textbf{Calculer la matrice inverse}
          \begin{itemize}
              \item[] Soit $A \in M_{n \times n}(\mathbb{C})$, alors
                    \begin{equation*}
                        A^{-1} = \frac{1}{\det A} \text{ adj}(A)
                    \end{equation*}
              \item[] $\text{adj}(A)$ est la matrice adjointe de $A$ et est définie comme \begin{equation*}
                        \text{adj}(A) = \left( \left( c_{ij} \right)_{n \times n} \right)^T
                    \end{equation*}
              \item[] $c_{ij}$ est le cofacteur d'incice $ij$ de $A$ et est défini comme \begin{equation*}
                        c_{ij} = (-1)^{i + j}m_{ij}
                    \end{equation*}
              \item[] $m_{ij}$ est le déterminant de la sous-matrice obtenue en biffant la ligne $i$ et la colonne $j$.
              \item[] On peut donc remarquer que $\det A = \sum_{k = 1}^{n} = a_{jk} c_{jk}$
              \item[] On a donc ce résultat pour les matrices inverses, \begin{equation*}
                        A \text{ est inversible} \iff \det A \neq 0
                    \end{equation*}
          \end{itemize}
\end{itemize}

\subsection*{Systèmes d'équations linéaires}
\begin{itemize}
    \item[] Chaque système d'équation linéaire peut-être représenter par une equation matricielle, soit \begin{equation*}
              \begin{cases}
                  a_{11}x_1 + a_{12}x_2 + \ldots + a_{1n}x_n & = b_1 \\
                  a_{21}x_1 + a_{22}x_2 + \ldots + a_{2n}x_n & = b_2 \\
                  \vdots                                             \\
                  a_{m1}x_1 + a_{m2}x_2 + \ldots + a_{mn}x_n & = b_m
              \end{cases} \iff \begin{pmatrix}
                  a_{11} & a_{12}      & \dots  & a_{1n} \\
                  a_{21} & a_{22}      & \dots  & a_{2n} \\
                  \vdots & \phantom{a} & \ddots & \vdots \\
                  a_{m1} & a_{m2}      & \dots  & a_{mn}
              \end{pmatrix} \begin{pmatrix}
                  x_1    \\
                  x_2    \\
                  \vdots \\
                  x_n
              \end{pmatrix} = \begin{pmatrix}
                  b_1    \\
                  b_2    \\
                  \vdots \\
                  b_m
              \end{pmatrix}
              \end{equation*}
    \item[] Le système devient $AX = B \iff (A|B)$. $(A|B)$ est dit $A$ augmenté de $B$ et est la matrice augmenté du système.
    \item[] \textbf{Forme échelonné d'un système} \begin{itemize}
              \item[] Une matrice est de la forme échelonné si \begin{enumerate}
                        \item La ligne qui précède une ligne non-nulle est non-nulle
                        \item Le premier coefficient non-nul d'une ligne non-nulle, appellé le pivot, est plus à
                              gauche que le premier coefficient non-nul de la ligne suivante.
                    \end{enumerate}
              \item[] On peut simplifier le système si on obtient la forme échelonné de la matrice augmenté. On obtient la forme échelonné
                    en appliquer des opérations élémentaires sur les lignes de $A$ et $B$. Ces opérations sont
                    échanger deux lignes, $L_j \leftrightarrow L_k$, additionner le multiple d'une ligne à une autre, 
                    $L_j \mapsto L_j + \alpha L_k$ et multiplier une ligne par un scalaire non-nul, $L_j \mapsto \alpha L_j, \ \alpha \neq 0$.
          \end{itemize}
    \item[] \textbf{Rang d'une matrice} \begin{itemize}
              \item[] $\text{rg}(A)$ est le rang de $A$ et est le nombre de ligne non-nulle de la forme échelonné
                    de $A$. On a alors trois cas pour un système linéaire $AX = B$, soit \begin{enumerate}
                        \item Si $\text{rg}(A) < rg(A|B)$, alors le système n'a pas de solution, il est incompatible
                        \item Si $\text{rg}(A) = rg(A|B) = n$, alors le système a une unique solution, il est compatible
                        \item Si $\text{rg}(A) = rg(A|B) < n$, alors le système a une infinité de solution, il est compatible
                    \end{enumerate}
              \item[] Dans le dernier cas, le système a une infinité de solution puisque qu'il a des variables libres,
                    soit exactement $n - \text{rg}(A)$ variables libres.
              \item[] Remarquons que si $A^\prime$ est la forme échelonné de $A$, alors $\det A = \alpha \det A^\prime, \ \alpha \in \mathbb{C}$.
                    Cela nous permet d'arriver au résultat \begin{equation*}
                        \det A \neq 0 \iff \text{rg}(A) = n \iff A \text{ est inversible} \iff AX=B \text{ a une unique solution}
                    \end{equation*}
              \item[] L'unique solution dans ce cas est $X = A^{-1}B$, puisque $A$ est inversible.
          \end{itemize}
    \item[] \textbf{Calculer l'inverse d'une matrice} \begin{itemize}
              \item[] Il est possible de calculer l'inverse de $A$ en échelonnant le système $(A|I)$. Le
                    système échelonné va donner $(A|I) \sim (I|B)$, avec $B = A^{-1}$.
          \end{itemize}
    \item[] \textbf{Système homogène} \begin{itemize}
              \item[] Un système homogène est un système de la forme $AX = \mathbb{O}$. Un tel système
                    est toujours compatible avec la solution trivial $X = \left(\begin{smallmatrix}
                                0 \\
                                \ldots \\
                                0
                            \end{smallmatrix}\right)$. De plus, si $A$ est inversible le système
                    homogène à seulement la solution trivial, sinon il a une infinité de
                    solution, puisque c'est impossible que $\text{rg}(A) < \text{rg}(A|\left(\begin{smallmatrix}
                                0 \\
                                \ldots \\
                                0
                            \end{smallmatrix}\right))$, et donc que le système soit incompatible.
          \end{itemize}
    \item[] \textbf{Noyau d'un système} \begin{itemize}
              \item[] Soit $A \in M_{m \times n}(\mathbb{C})$. Le noyau de $A$, noté $\text{N}(A)$, est l'ensemble de tous les solutions du système homogène 
            $AX = \mathbb{O}$, soit \begin{equation*}
                        \text{N}(A) = \left\{ X \in M_{n \times 1}(\mathbb{C}) \ | \ AX = \mathbb{O}  \right\}
                    \end{equation*}
          \end{itemize}
\end{itemize}
\section{Chapitre 3: Espace vectorielle de dimension finie}
\begin{definition}
      Un ensemble $V$ est un espace vectorielle sur un corps $F$
      si \begin{enumerate}[1)]
            \item $V$ est fermé sous l'addition c-à-d, $
                        \forall \ (v_1, v_2 \in V), \ v_1 + v_2 \in V
                  $
            \item $V$ est fermé sous la multication par un scalaire c-à-d, $
                        \forall \ (v \in V, \ \alpha \in F), \ \alpha v \in V
                  $
      \end{enumerate}
      Les éléments de $V$ sont appellés \guillemetleft \ vecteurs \guillemetright.
\end{definition}
\begin{remark}
      L'espace vectorielle le plus simple est $V = \{ 0_v \}$, ou $0_v$ est l'élément nulle.
\end{remark}
\begin{definition}
      Un scalaire $\alpha$ est un élément du corps $F$ associé à l'espace vectorielle.
\end{definition}
\noindent
Dans notre cas, ce corps est soit les nombres réels $\mathbb{R}$, ou les nombres complexes $\mathbb{C}$.

\subsection{Base d'un espace vectorielle}
\noindent
Soit un espace vectorielle $V$ et $\alpha_1, \alpha_2, \ldots, \alpha_n$ des scalaires.

\subsubsection{Combinaison linéaire}
\noindent
Soit $v_1, v_2, \ldots, v_n \in V$
\begin{definition}
      $\alpha_1 v_1 + \alpha_2 v_2 + \ldots + \alpha_n v_n$ est une combinaison linéaire de
      $v_1, v_2, \ldots, v_n$
\end{definition}

\subsubsection{Ensemble générateur}
\begin{definition}
      Un ensemble $S = \{ u_1, u_2, \ldots, u_n \} \subset V$, est un ensemble générateur si
      \[ \forall v \in V  \ \exists \alpha_1, \ldots, \alpha_n \text{ scalaire t.q. } v = \alpha_1 u_1 + \ldots + \alpha_n u_n\]
\end{definition}
\noindent
Note: Pour prouver qu'un ensemble est un ensemble générateur, il faut habituellement
prendre un élément général de l'espace vectorielle et exprimer cet élément général
comme une combinaison linéaire des vecteurs dans $S$.

\subsubsection{Indépendance linéaire}
\begin{definition}
      Un ensemble $S = \{ u_1, u_2, \ldots, u_n \} \subset V$, est un linéaire indépendant si
      \[
            \alpha_1 u_1 + \alpha_2 u_2 + \ldots + \alpha_n u_n = 0_v \implies \alpha_1 = \alpha_2 = \ldots = \alpha_n = 0
      \]
      sinon $S$ est linéairement dépendant, ou lié.
\end{definition}

\subsubsection{Déterminer l'indépendance linéaire de \texorpdfstring{$S \subset \R^n$}{S subset of Rn}}
\noindent
Soit $S = \{ u_1, u_2, \ldots, u_m \} \subset \R^n$ avec $u_j = \begin{pmatrix}
            u_{1j} & u_{2j} & \dots  & u_{nj} \end{pmatrix}^T, \ u_{kj} \in \R$ et $|S| = m$ \\
Considérons la matrice $M = \begin{pmatrix} u_1 & u_2 & \ldots & u_m \end{pmatrix}$. \\
Pour déterminer l'indépendance linéaire de $S$ à partir de $M$, il faut
échelonner la matrice $M$ ou la matrice $M^T$. En échelonnant $M$ ou $M^T$, on peut arriver 
à deux conclusions, soit \begin{enumerate}[1.]
      \item $\text{rg}(M^T) < m$ ou $\text{rg}(M) < m \implies S$ est lié.
      \item $\text{rg}(M^T) = m$ ou $\text{rg}(M^T) = m \implies S$ est linéairement indépendant.
\end{enumerate}
Note: Échelonner $M^T$ donne des résultats plus simple à interpréter puisque 
$0 \leq \text{rg}(M^T) \leq m$. Alors si $M^T$ échelonné contient une ligne nulle
on sait automatiquement que $S$ est lié, ce qui n'est pas le cas si $M$ échelonné
contient une ligne nulle.

\subsubsection{Base d'un espace vectorielle}
\begin{definition}
      Un ensemble $B = \{ u_1, u_2, \ldots, u_n \} \subset V$, est une base de $V$ si
      $B$ est un ensemble générateur de $V$ et $B$ est linéairement indépendant.
\end{definition}
\begin{definition}
      La dimension de $V$ est $\dim_F V = |B|$, ou $|B|$ est le nombre d'élément dans la base $B$
      de $V$ et $F$ est le corps de l'espace vectorielle.
\end{definition}
\begin{theorem}
      Le nombre de vecteur dans une base de $V$ ne dépend pas de la base choisi.
\end{theorem}
\begin{remark}
      Le théorème qui précède nous permet de définir la dimension de $V$ comme étant le
      nombre de vecteurs dans une base de $V$, puisque toutes les bases de $V$ contiennent
      le même nombre de vecteurs.
\end{remark}
\paragraph{Base canonique:}
Un espace vectorielle a souvent une base canonique, soit une base plus naturel à utiliser.
Par exemple, la base canonique de $\R^n$ est $S = \{ e_1, e_2, \ldots, e_n \}$.
On peut donc dire que $\dim_\R \R^n = n$

\subsubsection{Représentation d'un vecteur dans une base}
\noindent
Soit $V$ un espace vectorielle avec $\dim_F V = n$, et $B = \{ u_1, u_2, \ldots, u_n \}$ une base de $V$. \\
Cela veut dire que $ \forall \ v \in V  \ \exists \ \alpha_1, \ldots, \alpha_n \text{ scalaire t.q. } v = \alpha_1 u_1 + \ldots + \alpha_n u_n$
\begin{definition}
      Les scalaires $\alpha_1, \ldots, \alpha_n$ sont appellés les coordonnées de $v$ dans la base $B$.
\end{definition}
\begin{definition}
      La représentation de $v$ dans la base $B$ est noté $[v]_B$ et est exprimé
      \[
            [v]_B = \begin{pmatrix}
                  \alpha_1 \\
                  \alpha_2 \\
                  \vdots   \\
                  \alpha_n
            \end{pmatrix} \in F^n
      \]
\end{definition}

\subsubsection{Propriétés d'une base d'un espace vectorielle}
\begin{enumerate}[a)]
      \item Soit $v_1, v_2 \in V$, alors $v_1 = v_2 \iff [v_1]_B = [v_2]_B$
      \item Soit $S = \{ u_1, u_2, \ldots, u_n \} \subset V$ \\
            Si $S^\prime = \{ [u_1]_B, [u_2]_B, \ldots, [u_n]_B \}$ est linéairement indépendant alors $S$ l'est aussi
      \item La représentation dans une base est linaire, ce qui veut dire qu'elle satisfait ces deux propriétés:
            \begin{enumerate}[1.]
                  \item $[v_1 + v_2]_B = [v_1]_B + [v_2]_B$
                  \item $[\alpha v]_B = \alpha [v]_B$
            \end{enumerate}
\end{enumerate}
\section{Valeurs et vecteurs propres d'une matrice carré}

\subsection{Valeurs et vecteurs propres}
Soit $A \in M_{n \times n}(F)$
\begin{definition}
    $\lambda \in F$ valeur propre de $A \iff \exists \ u \in M_{n \times 1}(F), u \neq 0$ t.q. $Au = \lambda u$ \\
    Dans ce cas $u$ est dit le vecteur propre de $A$ associé à $\lambda$
\end{definition}
\begin{remark}
    Un vecteur propre est jamais nulle, mais une valeur propre peux être nulle
\end{remark}
\begin{definition}
    Le polynôme caractéristique de $A$ est $\chi_A(\lambda) = \det(A - \lambda I)$
\end{definition}
\begin{theorem}
    Les valeurs propres de $A$ sont les racines du polynôme caractéristique de $A$, c-à-d
    \[
        \lambda_0 \ \text{valeur propre de} \ A \iff \chi_A(\lambda_0) = 0
    \]
\end{theorem}
\subsubsection{Comment calculer les vecteurs propres}
\begin{enumerate}
    \item Trouver les racines de $\chi_A(\lambda)$
    \item Trouver toutes les solutions linéairement indépendantes du système homogène $(A - \lambda_0 I)X = \mathbb{O}$
          pour tout $\lambda_0$ valeurs propres de $A$.
    \item[] Note: Il se peut qu'une valeur propre ait plusieurs vecteurs propres linéairement indépendants.
\end{enumerate}
\subsubsection{Multiplicités et espace propre}
Grâce au théorème fondamentale de l'algèbre [\ref{theorem_fondamental_algebra}] on peut
exprimer le polynôme caractéristique à l'aide de ces racines, qui sont les valeurs propres
de $A$. Posons que $A$ a $m$ uniques valeurs propres. Alors
$\chi_A(\lambda) = (-1)^n(\lambda - \lambda_1)^{k_1}(\lambda - \lambda_2)^{k_2} \dots (\lambda - \lambda_m)^{k_m}
    \quad k_1 + k_2 + \dots + k_m = n, \ k_j \in \N$
\begin{definition}
    La multiplicité algébrique de la valeur propre $\lambda_j$ est le $k_j$ associé à $\lambda_j$
\end{definition}
\begin{definition}
    L'espace propre associé à $\lambda_j$ est
    $L_{\lambda_j} = \left\{ v \in M_{n \times 1} \mid Av = \lambda_j v \right\} \cup \left\{\mathbb{O} \in M_{n \times 1}\right\}$
\end{definition}
\begin{definition}
    La dimension de l'espace propre $L_{\lambda_j}$ est la multiplicité géométrique de $\lambda_j$
\end{definition}
\begin{lemma}
    $1 \leq \text{multiplicité géométrique de} \ \lambda_j \leq \text{multiplicité algébrique de} \ \lambda_j \leq n$
\end{lemma}

\subsection{Propriétés des valeurs propres}
Soit $A \in M_{n \times n}(\C)$
\paragraph{Propriété 1:} Si $B = \{v_1, \dots, v_n\}$ est un ensemble de vecteurs propres de $A$
associés à des valeurs propres distincts alors $B$ est linéairement indépendant.
\paragraph{Propriété 2:} Si $\lambda_0$ est une valeur propre de $A$ alors $(\lambda_0)^k$ est une
valeur propre de $A^k$, $k \in \N$
\paragraph{Propriété 3:} Si $\lambda_0$ est une valeur propre de $A$ et $A$ est inversible alors
$\frac{1}{\lambda_0}$ est une valeur propre de $A^{-1}$
\paragraph{Propriété 4:} Si $\det A = 0$ alors $\lambda_0 = 0$ est une valeur propre de $A$
\paragraph{Propriété 5:} $\det A = \lambda_1 \dots \lambda_n$ ou $\lambda_j$ est une valeur propre de $A$
\paragraph{Propriété 6:} $\text{tr} A = \lambda_1 + \dots + \lambda_n$ ou $\lambda_j$ est une valeur propre de $A$
\paragraph{Propriété 7:} $A$ et $A^T$ ont les même valeurs propres.
\paragraph{Propriété 8:} Si $A$ est triangulaire alors les valeurs propres de $A$ sont sa diagonale.
\paragraph{Propriété 9:} Les valeurs propres d'une matrice hermitienne sont réels
\paragraph{Propriété 10:} Si $A$ est hermitienne et que $v_1, v_2$ sont deux vecteurs propres de
$A$ associés à deux valeurs propres de $A$ distincts, soit $\lambda_1$ et $\lambda_2$, alors $v_1 \perp v_2$
et $L_{\lambda_1} \perp L_{\lambda_2}$ par rapport au produit scalaire canonique dans $\C^n$
\paragraph{Propriété 11:} \label{property_11} Si $A$ hermitienne, alors les vecteurs propres de $A$ forment une base orthonormale de $\C^n$

\subsection{Diagonalisation d'une matrice}
\begin{definition}
    Soit $A, P \in M_{n \times n}$ t.q. $P$ inversible, alors $A$ et $P^{-1}AP$ (ou $PAP^{-1}$) sont appellé semblabes
\end{definition}
\begin{remark}
    $\det(P^{-1}AP) = \det(PAP^{-1}) = \det A$ \\
    De plus, les valeurs propres de $A$ et $P^{-1}AP$ coïncident.
\end{remark}
\begin{definition}
    $A \in M_{n \times n}$ est diagonalisable $\iff \exists \ P \in M_{n \times n}$ inversible t.q. $P^{-1}AP$ diagonale
\end{definition}
\begin{theorem}
    $A \in M_{n \times n}$ est diagonalisable $\iff A$ admet $n$ vecteurs propres linéairement indépendant \\
    Dans ce cas on a que \[P = \begin{bmatrix}
            v_1 & v_2 & \dots & v_n
        \end{bmatrix} \ \text{avec} \ v_1, v_2, \dots, v_n \ \text{les vecteurs propres de} \ A\] 
        De plus, on a que
        \[P^{-1}AP = \text{diag}\{\lambda_1, \dots, \lambda_n\} \ \text{avec} \ \lambda_1, \dots, \lambda_n 
        \ \text{les valeurs propres de} \ A\] 
\end{theorem}
\begin{corollary}
    Si $A$ a $n$ valeurs propres distincts, alors $A$ est diagonalisable
\end{corollary}
\begin{corollary}
    Une matrice hermitienne est toujours diagonalisable à cause de \ref{property_11}
\end{corollary}
\begin{corollary}
    \label{hermitienne_implies_unitaire_P}
    Si $A$ hermitienne alors $\exists \ P$ unitaire t.q. $P^{-1}AP = P^\dagger A P$ diagonale
\end{corollary}
\begin{remark}
    La preuve de \ref{hermitienne_implies_unitaire_P} utilise cette équivalence, soit
    \[
        \begin{matrix}
            P \ \text{unitaire} \iff & \text{les colonnes de} \ P \ \text{forment une base orthonormale dans} \\
                                     & \C^n \ \text{par rapport au produit scalaire canonique}
        \end{matrix}
    \]
\end{remark}

\subsection{Fonctions d'une matrice}
Soit $f(x) \in \C_n[x]$, ou $\C_n[x]$ est l'espace des polynômes complexes à exposant au plus $n$
\begin{definition}
    Soit $A \in M_{n \times n}(\C)$, alors $f(A) = \alpha_0 I + \alpha_1 A + \dots + \alpha_n A^n$
\end{definition}
Un autre cas considéré dans ce cours est $f(z) = e^z = \sum_{k = 0}^{\infty}\frac{z^k}{k!}$
\begin{definition}
    Soit $A \in M_{n \times n}(\C)$, alors $e^A = \sum_{k = 0}^{\infty}\frac{A^k}{k!} = I + A + \frac{A^2}{2!} + \frac{A^3}{3!} + \dots$
\end{definition}

\subsubsection{Comment calculer \texorpdfstring{$f(A)$}{f(A)}}
Il a quatre choses qui peut nous aider pour calculer $f(A)$
\begin{enumerate}
    \item $A$ est nilpotente d'incide $k$, c-à-d que $A^k = \mathbb{O}$, donc on doit seulement 
    calculer les $k + 1$ premiers termes de $f(A)$
    \item Il existe une formule pour les puissances de $A$ (souvent prouver par récurrence),
    alors il est possible d'utiliser cette formule pour simplifier le calcul
    \item $A = \text{diag}\{\lambda_1, \dots, \lambda_n \} \implies f(A) = \text{diag}\{f(\lambda_1), \dots, f(\lambda_n)\}$
    \item Si $A$ est diagonalisable alors $A = PDP^{-1}$ avec $D = \text{diag}\{\lambda_1, \dots, \lambda_n\}$ \\
    Alors $f(A) = Pf(D)P^{-1} = P\text{diag}\{f(\lambda_1), \dots, f(\lambda_n)\}P^{-1}$
\end{enumerate}
\begin{remark}
    Si $A$ est non diagonalisable et qu'il n'existe pas de formule pour les puissances de $A$,
    il faut calculer tout les termes de $f(A)$ pour obtenir la réponse. 
\end{remark}

\subsubsection{Propriétés de l'exponentielle d'une matrice}
\begin{enumerate}
    \item Comme $A$ commute avec $A^k \ \forall \ k \in \N$ alors $A$ commute avec $e^A$
    \item Si $AB = BA$, soit que $A$ et $B$ commute alors $e^{A + B} = e^A e^B$
    \item $e^A$ inversible $\forall \ A \in M_{n \times n}$ avec $\left(e^A\right)^{-1} = e^{-A}$ 
    \item Si $A$ est hermitienne alors $e^{iA}$ est unitaire
\end{enumerate}
\unnumsec{Espace dual d'un espace vectoriel}

\subsection{Fonctionnelles linéaires}
Soit $V$ un espace vectoriel avec les scalaires $F$
\begin{definition}
    Une forme linéaire, ou une fonctionnelle linéaire, est une application
    $f\colon V \to F$ qui, pour tout $u, v \in V$, $\scalaire{\alpha}$,
    satisfait les deux propriétés suivantes:
    \begin{align*}
        1. \quad f(u + v) = f(u) + f(v)& &2. \quad f(\alpha v) = \alpha f(v)
    \end{align*}
\end{definition}
\begin{remark}
    La fonctionnelle linéaire la plus simple est $\myfunc{f}{V}{F}{v}{0}$
\end{remark}
\begin{definition}
    Soit $f_1, f_2$ deux fonctionnelles linéaires. \\
    Définissons leur somme $f_1 + f_2$ comme telle:
    $(f_1 + f_2)(v) = f_1(v) + f_2(v) \ \forall \ v \in V $ \\
    Définissons la multiplication par un scalaire $\alpha f_1$ comme telle: 
    $(\alpha f_1)(v) = \alpha f_1(v) \ \forall \ v \in V$
\end{definition}
\begin{lemma}
    L'ensemble des fonctionnelles linéaires sur $V$ est lui-même un espace vectoriel
\end{lemma}
\begin{definition}
    L'espace des fonctionnelles sur $V$ est appelé l'espace dual de $V$ et est noté $V^*$
\end{definition}
\begin{theorem}
    Soit $B = \{b_1, \dots, b_n\}$ une base dans $V$ avec $[v]_B = \begin{pmatrix}
        v_1 & \dots & v_n \end{pmatrix}^T \ v \in V$. 
    Alors la base de $V^*$ est $B^* = \{\epsilon_1, \dots, \epsilon_n\}$ où 
    $\myfunc{\epsilon_j}{V}{F}{v}{v_j}$, et $v_j$ est la j-ème coordonnée de $v$ dans la base $B$.
\end{theorem}
\begin{remark}
    La définition de $\epsilon$ implique que $\epsilon_j(b_k) = \delta_{jk} \ \forall \ b_k \in B$
\end{remark}
\begin{corollary}
    $\dim V = n \implies \dim V^* = n$
\end{corollary}
\begin{corollary}
    $\left(\C^n\right)^* = \C^n$ et $\left(\R^n\right)^* = \R^n$ 
\end{corollary}

\subsection{Espace dual dans l'espace vectoriel muni d'un produit scalaire}
Soit $V$ un espace vectoriel muni d'un produit scalaire $\scpr{\cdot}{\cdot}$
\begin{definition}
    Le vecteur dual de $x \in V$ est un élément de l'espace dual $V^*$ notée $f_x$ ou $\hat{x}$. $f_x$ est la fonctionnelle linéaire qui projette les vecteurs sur $x$, c.-à-d.
    \[ \myfunc{f_x}{V}{F}{v}{\scpr{x}{v}} \]
\end{definition}
\begin{theorem} Chaque vecteur de $V$ a un unique vecteur dual, c.-à-d.
    \[\forall \ f \in V^* \ \exists! \ x \in V \ \text{t.q.} \ f = f_x\]
\end{theorem}
\subsubsection{Comment trouver le vecteur dual d'un élément de l'espace dual}
Soit $B = \{b_1, \dots, b_n\}$ une base \underline{orthonormale par rapport à $\scpr{\cdot}{\cdot}_V$} dans $V$ et $f \in V^*$ \\
Posons $f(b_j) = a_j \in \C^n$, alors la représentation du vecteur dual $x$ à $f$ dans la base $B$ est donnée par 
\[
[x]_B = \begin{pmatrix} (a_1)^* & \dots & (a_n)^* \end{pmatrix}^T 
\]
\begin{note}
    Dans le cas où le produit scalaire
    dans $V$ n'est pas le produit scalaire canonique, il est souvent plus simple d'utiliser une 
    méthode plus directe que de trouver une base orthonormale par rapport au produit scalaire dans $V$.
\end{note}
\begin{definition}
    Soit $B = \{u_1, \dots, u_n\}$ une base dans $V$, alors 
    une base $B^* = \{\epsilon_1, \dots, \epsilon_n\}$ dans $V^*$ est dite base dual à 
    $B \iff \epsilon_j(u_k) = \delta_{jk}$ 
\end{definition}
\subsection{Notation de Dirac}
Soit $V$ un espace vectoriel muni d'un produit scalaire.
\begin{definition}
    Dans la notation de Dirac, un élément $v \in V$ est noté $\ket{v} \in V$ et $\ket{\cdot}$ est appelé un ket. 
    Le vecteur dual de $\ket{v}$ est noté $\bra{v}$ et $\bra{\cdot}$ est appelé un bra. L'évaluation
    du vecteur dual $f_v = \ket{v}$ avec un vecteur $x \in V$ est notée $f_v(x) = f_v(\ket{x}) = \braket{v}{x}$
\end{definition}
\unnumsec{Chapitre 6: Opérateurs linéaires}
\begin{definition}
    Un opérateur $T$ est une application $T\colon V \to W$ avec $V$ et $W$ des espaces vectoriels
\end{definition}
\begin{definition}
    Un opérateur $T$ est linéaire $\iff T(\alpha u + \beta v) = \alpha T(u) + \beta T(v) \ \forall \ u, v \in V, \ \alpha, \scalaire{\beta}$ 
\end{definition}
\begin{definition}
    Soit $T_1 \colon V \to W$ et $T_2 \colon W \to U$. La composition de $T_1$ et $T_2$ est noté
    $T_2 T_1$ et l'évaluation est $T_2 T_1 (v) = T_2(T_1(v))$ avec $T_1 T_2 \colon V \to U$
\end{definition}
\begin{remark}
    La composition de deux opérateurs n'est généralement pas commutatif, mais est associatif
\end{remark}
\begin{definition}
    Soit $T_1, T_2 \colon V \to V$. Le commutateur de $T_1$ et $T_2$ est 
    $[T_1, T_2] = T_1T_2 - T_2T_1$ avec $[T_1, T_2](v) = T_1T_2v - T_2T_1v$
\end{definition}

\subsection{La matrice d'un opérateurs linéaires}
\begin{theorem}
    Soit $V, W$ deux espaces vectoriels avec $\dim_\C V = n$ et $\dim_\C W = m$ \\
    Soit $B_V, B_W$ deux bases respectives de $V$ et $W$, alors 
    \[
    T\colon V \to W \ \text{est linéaire} \iff \exists \ M \in M_{m \times n}(\C) \ \text{t.q.} \ 
    \forall \ v \in V \ M[v]_{B_V} = [Tv]_{B_W}
    \]  
    De plus, si $B_V = \{v_1, \dots, v_n\}$ alors la matrice $M$ est donnée par 
    \[
    M = \left[ \left[ Tv_1 \right]_{B_W}, \dots, \left[ Tv_n \right]_{B_W} \right]
    \]
\end{theorem}
\begin{definition}
    La matrice $M$ t.q. $M[v]_{B_V} = [Tv]_{B_W}$ est dite la matrice de l'opérateur $T$
    dans les deux base $B_V, B_W$, notée $M = [T]_{B_V, B_W}$. Si $V = W$ et $B_V = B_W$
    alors on note $M = [T]_{B_V}$
\end{definition}

\subsection{Représentation d'un opérateur linéaire dans une nouvelle base}
\begin{theorem}
    Soit $T\colon V \to W$ un opérateur linéaire. \\
    Soit $B_V, B_V^\prime$ deux bases dans $V$ et $B_W, B_W^\prime$ deux bases dans $W$, alors
    $[T]_{B_V^\prime, B_W^\prime} = P_{B_W^\prime}^{B_W} [T]_{B_V, B_W} P^{B_V^\prime}_{B_V}$ \\
    Si $V = W$ et $B_V = B_W$, alors $[T]_{B_V} = P_{B_V^\prime}^{B_V} [T]_{B_V} P^{B_V^\prime}_{B_V}$
\end{theorem}
\begin{theorem}
    Soit $T_1, T_2 \colon V \to V$ des opérateurs linéaires, $B_V$ une base dans $V$ et $\scalaire{\alpha}$, alors 
    les opérateurs suivant sont linéaires, soit $T_1 + T_2, \ \alpha T_1, \ T_1T_2, \ [T_1, T_2]$.
    Les matrices des opérateurs suivants sont 
    \begin{align*}
        1) \quad &[T_1 + T_2]_{B_V} = [T_1]_{B_V} + [T_2]_{B_V}& &2) \quad [\alpha T_1]_{B_V} = \alpha [T_1]_{B_V} \\
        3) \quad &[T_1T_2]_{B_V} = [T_1]_{B_V}[T_2]_{B_V}& &4) \quad \left[[T_1, T_2]\right]_{B_V} = \left[[T_1]_{B_V}, [T_2]_{B_V}\right]
    \end{align*}
\end{theorem}

\subsection{Valeurs et vecteurs propres d'un opérateur}
\begin{definition}
    Soit $T\colon V \to V$ un opérateur linéaire, alors 
    \[v \in V \ \text{vecteur propre de} \ T \iff \exists \ \scalaire{\lambda} \ \text{t.q.} \ Tv = \lambda v\]
    $\lambda$ est appelé la valeur propre de $T$
\end{definition}
\begin{theorem}
    Si $\dim V \in \N$, alors $\forall \ v \in V$ vecteur propre de $T$, $[v]_B$ vecteur propre de $[T]_B$ \\
    Cela va de même pour les valeurs propres de $T$
\end{theorem}
\begin{definition}
    L'ensemble des valeurs propres d'un opérateur s'appelle le spectre de l'opérateur
\end{definition}
\begin{theorem}
   $\text{Ker}(T) \neq \{0_V\} \iff \lambda = 0$ valeur propre de $T$, ou $\text{Ker}(T) = \left\{ v \in V \mid Tv = 0_W \right\}$
\end{theorem}
\unnumsec{Chapitre 7: Opérateurs inverses, hermitiens et unitaires}

\subsection{Opérateurs inverses}
Soit $T\colon V \to W$ un opérateur linéaire
\begin{definition}
    $T$ inversible $\iff \forall \ w \in W \ \exists! \ v \in V$ t.q. $Tv = w$
\end{definition}
\begin{definition}
    L'opérateur inverse de $T$ est $\myfunc{T^{-1}}{W}{V}{w}{v}$ 
\end{definition}
Soit $B_V, B_W$ des bases dans $V, W$ respectivement 
\begin{theorem}
    $T$ inversible $\iff \dim V = \dim W$ et $[T]_{B_V, B_W}$ inversible \\
    Dans ce cas, $T^{-1} = [T^{-1}]_{B_W, B_V} = \left([T]_{B_V, B_W}\right)^{-1}$
\end{theorem}
\begin{corollary}
    $T$ inversible $\iff [T]_{B_V, B_W}$ carré
\end{corollary}
\begin{remark}
    $[T]_{B_V, B_W}$ carré $\implies \dim V = \dim W$
\end{remark}

\subsubsection{Propriétés de l'opérateur inverse}
Soit $T, S\colon V \to W$ des opérateurs linéaires inversibles
\begin{enumerate}
    \item $T^{-1}\colon W \to V$ est linéaire
    \item $[T^{-1}]_{B_W, B_V} = \left([T]_{B_V, B_W}\right)^{-1}$
    \item $\left(T^{-1}\right)^{-1} = T$
    \item $(TS)^{-1} = S^{-1}T^{-1}$
    \item $T$ inversible $\iff \text{Ker}(T) = \{0_W\}$
\end{enumerate}

\subsection{Opérateurs hermitiens}
Soit $V$ un espace vectoriel muni d'un produit scalaire $\scpr{\cdot}{\cdot}$ \\
Soit $B$ une base orthonormale dans $V$ par rapport au produit scalaire dans $V$ \\
Soit $A\colon V \to V$ un opérateur linéaire
\begin{definition}
    $A$ hermitien $\iff [A]_B$ hermitien 
\end{definition}
\begin{theorem}
    Peut importe la base $B$ choisi, si $A$ hermitien alors $[A]_B$ hermitien
\end{theorem}
\begin{remark}
    Le théorème nous permet de justifier la définition d'un opérateur hermitien, puisque 
    le théorème prouve que la définition ne dépend pas de la base choisi
\end{remark}

\subsubsection{Propriétés de l'opérateur hermitien}
Soit $A, T\colon V \to V$. Les opérateurs suivants sont hermitiens
\begin{align*}
    1. \quad A + T& &2. \quad \alpha A \ \text{si} \ \alpha \in \R& &3. \quad AT \ \text{si} \ A,T \ \text{commute}& &4. \quad A^n, T^n& &5. \quad A^{-1} \ \text{si} \ A \ \text{inversible} 
\end{align*}
De plus, un opérateur hermitien est indépendant dans le produit scalaire. 
\begin{lemma}
    Soit $v_1, v_2 \in V$, alors $\scpr{v_1}{Av_2}_V = \scpr{Av_1}{v_2}_V$
\end{lemma}
\noindent
Dans la notation de Dirac, on note cette propriété comme tel: $\bra{v_1}A\ket{v_2}$,
ce qui indique que $A$ peut être appliqué à $\bra{v_1}$ ou $\ket{v_2}$

\subsection{Opérateur unitaire}
Soit $V$ un espace vectoriel muni d'un produit scalaire $\scpr{\cdot}{\cdot}$ \\
Soit $B$ une base orthonormale dans $V$ par rapport au produit scalaire dans $V$ \\
Soit $A\colon V \to V$ un opérateur linéaire
\begin{definition}
    $A$ unitaire $\iff [A]_B$ unitaire 
\end{definition}
\begin{remark}
    Comme dans le cas de l'opérateur hermitien, cette définition ne dépend pas de la base $B$ choisi
\end{remark}

\subsubsection{Propriétés de l'opérateur unitaire}
\begin{lemma}
    Si $A\colon V \to V$ unitaire et $v_1, v_2 \in V$, alors $\scpr{v_1}{v_2}_V = \scpr{Av_1}{Av_2}_V $
\end{lemma}

\end{document}
